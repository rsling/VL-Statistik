\section{Fisher-Exakt-Test und Stichprobengröße}

\begin{enumerate}\Lf
  \item Rekapitulieren Sie die Berechnung des klassischen \textit{Tea Tasting Lady}-Experiments für 6 richtige Tassen bei 8 Tassen insgesamt auf den Folien.
  \item Berechnen Sie die Wahrscheinlichkeiten bzw.\ den p-Wert für dasselbe Verhältnis von richtigen Tassen, aber bei einer zehn Mal größeren Stichprobe, also 60 Tassen korrekt vorhergesagt bei 80 Tassen insgesamt.
  \item Interpretieren Sie das Ergebnis.
  \item Stellen Sie die Anfangswerte der zweiten Berechnung als Vier-Felder-Tabelle dar.
\end{enumerate}


\section{Vorgriff auf Kollostruktionsanalysen mit Fisher-Test}

Uns werden im weiteren Verlauf sogenannte \textit{Kollostruktionsanalysen} begegnen.
Diese sind ein Analyseverfahren für Korpusdaten und wurden ursprünglich (bei ihrer Erfindung vor 20 Jahren) mit Fisher-Exakt-Tests gerechnet.
Ein möglicher Datensatz wäre folgender:%
\footnote{Normalerweise nimmt man etwas andere Designs, aber das schauen wir uns dann in Ruhe an.}

\begin{center}
\begin{tabular}[h]{|l|c|c|}
  \cline{2-3}
  \multicolumn{1}{c|}{} & \textbf{\textit{kaufen}} & \textbf{\textit{umarmen}} \\\hline
  \textbf{im Passiv} & 120 & 30 \\\hline
  \textbf{im Aktiv} & 380 & 470 \\\hline
\end{tabular}
\end{center}

Konzeptionell sind die tabulierten Zahlen die Anzahlen der Vorkommnisse von (in diesem konkreten Fall) dem Verb \textit{kaufen} im Aktiv und im Passiv und dem Verb \textit{umarmen} -- ebenfalls im Aktiv und im Passiv -- in einer Stichprobe aus irgendeinem Korpus (= Textsammlung) des Deutschen.
(Die Zahlen sind hier der Einfachheit halber ausgedacht.)
Wir möchten mit so einer Untersuchung quantifizieren, wie hoch die Affinität \textbf{der Verben im Vergleich zueinander} zur Passivbildung ist.

\begin{enumerate}\Lf
  \item Wie viele Vorkommnisse von \textit{kaufen} und \textit{umarmen} haben wir jeweils beobachtet?
  \item Wie viele Sätze im Passiv und im Aktiv haben wir jeweils beobachtet?
  \item Wie groß war die gesamte Stichprobe?
  \item Was ist das Verhältnis zwischen Passiv und Aktiv bei \textit{kaufen} und bei \textit{umarmen}?
  \item Wie würden Sie den p-Wert berechnen? Es reicht der Lösungsweg.
  \item Der p-Wert ist $5,184\cdot 10^{-16}$. Überführen Sie diese Zahl aus der Exponentialschreibweise in herkömmliche Fließkommanotation, falls Sie das noch können.
  \item Was sagt uns dieser p-Wert? Was ist das für eine Wahrscheinlichkeit? Das ist ja zunächst mal von der Idee her etwas völlig anderes als bei der \textit{Tea Tasting Lady}.
  \item Es gibt zwischen den Designs der \textit{Tea Tasting Lady} und der Kollostruktionsanalyse einen wesentlichen Unterschied. Finden Sie den? Das ist allerdings eine optionale Transferaufgabe auf sehr hohem Niveau.
\end{enumerate}
