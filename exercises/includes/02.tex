\begin{center}
  Hinweis: Wo nicht anders angegeben, runden Sie die Ergebnisse auf zwei Nachkommastellen.
\end{center}

\section{Skalenniveaus}

Bestimmen Sie das Skalenniveau von folgenden Messgrößen:

\begin{enumerate}\Lf
  \item Prozentwerte \Sol{Verhältnisskala}
  \item Wortfrequenz-Rang (häufigstes Wort, \ldots, seltenstes Wort) \Sol{Ordinalskala}
  \item Kasus \Sol{Nominalskala}
  \item Geschwindigkeit \Sol{Intervallskala}
  \item Akzentsitz (Erstsilbe, Mittelsilbe, Endsilbe) \Sol{Nominalskala}
  \item Satzlänge, gemessen in Wörtern \Sol{Intervallskala}
  \item Frequenz eines Wortes im Korpus (absolute Zahl) \Sol{Intervallskala}
  \item Höhe über NN \Sol{Verhältnisskala}
  \item DSH-Prüfungsniveau (I -- III) \Sol{Ordinalskala}
  \item Verhältnis Satzlänge in Wörtern zu Wortlänge in Silben in einem Text \Sol{Verhältnisskala}
  \item Wortklasse (=~Wortart) \Sol{Nominalskala}
  \item Beschleunigung \Sol{Verhältnisskala}
  \item Textniveau (leicht, mittel, schwer) \Sol{Ordinalskala}
  \item Frequenz eines Wortes im Korpus pro eine Millionen Wörter \Sol{Intervallskala}
  \item Textsorte \Sol{Nominalskala}
\end{enumerate}

\section{Modus und Median}
\label{sec:modmed}

Ermitteln Sie den Modus und wo möglich den Median für folgende Messreihen von Hand (ohne Software):
% Zeichnen Sie Histogramme (Häufigkeitsdiagramme) und ermitteln Sie den Modus und wenn möglich den Median für folgende Messreihen:

\begin{enumerate}
  \item $x=[$Nom, Akk, Akk, Akk, Nom, Dat, Gen, Nom, Nom, Akk, Dat, Dat, Akk, Akk$]$\Sol{\\Modus: 6, Median: n.\,d.}
  \item $x=[$4, 5, 3, 3, 3, 2, 1, 2, 2, 1, 5, 4, 2, 2, 1, 3, 2$]$\Sol{\\Modus: 2, Median: 2}
  \item $x=[$4.3, 5.0, 3.0, 3.3, 3.7, 2.3, 1.3, 2.7, 2.0, 1.0, 5.0, 4.3, 2.0, 2.0, 1.3, 3.0, 2.7$]$\Sol{\\Modus: 2.0, Median: 2.7}
\end{enumerate}

\newpage

\section{Mittel und Streuung}
\label{sec:mittelstreu}

Ermitteln Sie von Hand für die untenstehenden Messreihen das arithmetische Mittel, die Varianz und die Standardabweichung:
% Zeichnen Sie Histogramme und ermitteln Sie zuerst von Hand und dann in \texttt{R} das arithmetische Mittel, die Varianz und die Standardabweichung für folgende Messreihen:

\begin{enumerate}
  \item $x=[2.73, 1.85, 21.24, 17.97, 5.49, 18.90, 12.46, 0.97, 6.45, 7.43]$\Sol{\\$\bar{x}=9.55$, $var(x)=57.05$, $stdev(x)=7.55$}
  \item $x=[1.00, 1.91, 3.12, 4.38, 4.72, 5.29, 3.82, 3.25, 2.04, 0.93]$\Sol{\\$\bar{x}=3.05$, $var(x)=2.36$, $stdev(x)=1.54$}
  \item $x=[1.07, 1.06, 0.94, 1.84, 3.04, 3.22, 4.18, 5.27, 6.27, 6.75]$\Sol{\\$\bar{x}=3.36$, $var(x)=4.79$, $stdev(x)=2.19$}
\end{enumerate}

\section{z-Werte und Standardfehler}

Ermitteln Sie für die Messreihen aus Aufgabe~\ref{sec:mittelstreu} die z-Werte für die Messpunkte und die Standardfehler von Hand.
Formulieren Sie in eigenen Worten (jeweils ein Satz), was z-Werte und Standardfehler angeben.

\Sol{\begin{enumerate}
  \item z-Werte: $[-0.9, -1.02, 1.55, 1.11, -0.54, 1.24, 0.39, -1.14, -0.41, -0.28]$, $SF=2.39$
  \item z-Werte: $[-1.33, -0.74, 0.05, 0.87, 1.09, 1.46, 0.5, 0.13, -0.66, -1.38]$, $SF=0.49$
  \item z-Werte: $[-1.05, -1.05, -1.11, -0.7, -0.15, -0.07, 0.37, 0.87, 1.33, 1.55]$, $SF=0.69$
  \end{enumerate}}

  \Sol{Die z-Werte sind die in Standardabweichungen normierten Abweichungen der Messwerte vom Mittelwert der Stichprobe. Der Standardfehler ist die mittlere Abweichung des beobachteten Stichprobenmittels vom wahren Populationsmittel in wiederholten Stichproben der gegebenen Größe (falls die Varianz in der Population der Varianz in der Stichprobe entspricht). (Die "`gegebene Größe"' ist die Größe der tatsächlichen Stichprobe.) Vereinfacht gesagt ist er der Erwartungswert, um den Stichproben der vorliegenden Größe vom wahren Mittelwert abweichen.}

% \section{Quartile}
% 
% Berechnen Sie für alle Messreihen aus Aufgabe~\ref{sec:modmed} und~\ref{sec:mittelstreu} das Minimum, die Quartile 1--4 und den Interquartilsabstand.
% Zeichnen Sie von Hand einen Boxplot für die Messreihen 1 und 3 aus Aufgabe~\ref{sec:mittelstreu}.
% Überprüfen Sie das Ergebnis mittels der Funktion \texttt{boxplot()} in \texttt{R}.

% \section{Kovarianz und Korrelationskoeffizient}
% 
% Zeichnen Sie die folgenden Messungen in ein Koordinatensystem, wobei $x_i$ und $y_i$ jeweils einen Messpunkt darstellen.
% Berechnen Sie zuerst von Hand und dann in \texttt{R}\footnote{\texttt{cov()} und \texttt{cor()}} die Kovarianz ($cov$) und den Korrelationskoeffizienten ($r$):
% 
% \begin{itemize}
%   \item[ ] $x=[4.52, 5.37, 4.28, 3.85, 3.23, 2.72, 1.83, 2.12, 1.04, 0.96]$
%   \item[ ] $y=[1.08, 2.12, 3.10, 4.26, 4.73, 6.13, 7.63, 7.83, 9.45, 9.60]$
% \end{itemize}
% 
% \textbf{Vertiefende Aufgabe 1}:
% Nehmen Sie in \texttt{R} die beiden Vektoren mal 100 und berechnen Sie Kovarianz und Korrelation erneut.
% Interpretieren Sie das Ergebnis.
% 
% \textbf{Vertiefende Aufgabe 2}:
% Erfinden Sie selber eine Messreihe (ggf.\ durch einen graphischen Ansatz), bei der $r$ möglichst nahe an $0$ ist.
% Berechnen Sie $r$ dafür.
% 
% \textbf{Vertiefende Aufgabe 3 (mit Lektüre)}:
% Finden Sie mit Gries, der R-Hilfe zu \texttt{cor}, Wikipedia usw.\ heraus, was \textit{Kendalls $\tau$} im Gegensatz zu \textit{Pearsons $r$}, das wir berechnet haben ist.
% Beides sind Korrelationskoeffizienten.
% Denken Sie sich selber Datenreihen mit jeweils einem hohen und einem niedrigen $\tau$ aus.

\section{Konfidenzintervalle (Anteilswerte)}

\subsection{Berechnung des Konfidenzintervalls für Anteilswerte}

Berechnen Sie für folgende Anteilswerte ($f$) die Konfidenzintervalle bei den Stichprobengrößen $n=10$ und $n=100$ auf den Konfidenzniveaus $0.9$ und und $0.99$ (also je vier Mal den unteren un oberen Wert des Konfidenzintervalls).
Die kritischen Werte der Normalverteilung entnehmen Sie bitte der zur Verfügung gestellten Tabelle.
 Nachkommastellen.Runden Sie auf drei Nachkommastellen.

\begin{enumerate}
  \item $f=0.21$ 
  \item $f=0.49$
  \item $f=0.89$
\end{enumerate}

\Sol{%
  \begin{center}
    \begin{tabular}[h]{rcccc}
      \hline
      \textbf{f} & \textbf{n=10, k=0.9} & \textbf{n=100, k=0.9} & \textbf{n=10, k=0.99} & \textbf{n=100, k=0.99} \\
      \hline
      0.21 & $[-0.002, 0.422]$ & $[0.143, 0.277]$ & $[-0.122, 0.542]$ & $[0.105, 0.315]$ \\
      0.49 & $[0.23, 0.75]$ & $[0.408, 0.572]$ & $[0.083, 0.897]$ & $[0.361, 0.619]$ \\
      0.89 & $[0.727, 1.053]$ & $[0.839, 0.941]$ & $[0.635, 1.145]$ & $[0.809, 0.971]$ \\
      \hline
    \end{tabular}
  \end{center}
}


\subsection{Schlechte Praxis beim Berichten von Konfidenzintervallen}

Warum hätte folgende Tabelle ganz ohne Nachrechnen nicht gedruckt werden dürfen?

\begin{center}
  \includegraphics[width=1\textwidth]{graphics/cischrott}
\end{center}
