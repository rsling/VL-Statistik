\usetikzlibrary{calc,decorations.pathmorphing,tikzmark,positioning,chains,trees,graphs,shapes,shadows,arrows}

\renewcommand\tikzmark[2]{%
  \tikz[remember picture,baseline=(chain-1.base),start chain] \node[on chain,inner sep=2pt,outer sep=0] (#1){#2};%
}

\newcommand\link[2]{%
  \begin{tikzpicture}[remember picture, overlay, >=stealth, shift={(0,0)}]
    \draw[-] (#1) --++(0,-12pt) -| (#2);
   \end{tikzpicture}%
}


\makeatletter
\g@addto@macro{\endtabular}{\rowfont{}}% Clear row font
\makeatother
\newcommand{\rowfonttype}{}% Current row font
\newcommand{\rowfont}[1]{% Set current row font
   \gdef\rowfonttype{#1}#1%
}
\newcolumntype{L}{>{\rowfonttype}l}


\usepackage{tikz-qtree}
\usepackage[linguistics]{forest}
\usepackage[maxbibnames=99,
  maxcitenames=2,
  uniquelist=false,
  backend=biber,
  doi=false,
  url=false,
  isbn=false,
  bibstyle=biblatex-sp-unified,
  citestyle=sp-authoryear-comp]{biblatex}

% Biblatex ============================================================

\addbibresource{rs.bib}

% Colors ==============================================================

\ifdefined\HANDOUT
  \definecolor{grau}{rgb}{0.5,0.5,0.5}
  \definecolor{lg}{rgb}{0.8,0.8,0.8}
  \definecolor{trueblue}{rgb}{0.3,0.3,1}
  \definecolor{ltb}{rgb}{0.8,0.8,1}
  \definecolor{lgr}{rgb}{0.5,1,0.5}
  \definecolor{orongsch}{RGB}{255,165,0}
  \definecolor{gruen}{rgb}{0,0.4,0}
  \definecolor{rot}{rgb}{0.7,0.2,0.0}
  \definecolor{tuerkis}{RGB}{63,136,143}
  \definecolor{braun}{RGB}{108,71,65}
  \definecolor{blaw}{rgb}{0,0,0.9}
\else
  \definecolor{grau}{rgb}{0.7,0.7,0.7}
  \definecolor{lg}{rgb}{0.9,0.9,0.9}
  \definecolor{trueblue}{rgb}{0.8,0.8,1}
  \definecolor{ltb}{rgb}{0.9,0.9,1}
  \definecolor{lgr}{rgb}{0.7,1,0.7}
  \definecolor{orongsch}{RGB}{255,200,100}
  \definecolor{gruen}{RGB}{0,230,0}
  \definecolor{rot}{RGB}{255,155,100}
  \definecolor{tuerkis}{RGB}{150,205,205}
  \definecolor{braun}{RGB}{140,120,115}
  \definecolor{blaw}{rgb}{0,0,0.9}
\fi

\newcommand{\gruen}[1]{\textcolor{gruen}{#1}}
\newcommand{\blaw}[1]{\textcolor{blaw}{#1}}
\newcommand{\rot}[1]{\textcolor{rot}{#1}}
\newcommand{\blau}[1]{\textcolor{trueblue}{#1}}
\newcommand{\orongsch}[1]{\textcolor{orongsch}{#1}}
\newcommand{\grau}[1]{\textcolor{grau}{#1}}
\newcommand{\whyte}[1]{\textcolor{white}{#1}}
\newcommand{\tuerkis}[1]{\textcolor{tuerkis}{#1}}
\newcommand{\braun}[1]{\textcolor{braun}{#1}}

% Newcommands =========================================================

\newcommand{\Dim}{\cellcolor{lg}}
\newcommand{\Dimblue}{\cellcolor{ltb}}
\newcommand{\Dimgreen}{\cellcolor{lgr}}
\newcommand{\Sub}[1]{\ensuremath{_{\text{#1}}}}
\newcommand{\Up}[1]{\ensuremath{^{\text{#1}}}}
\newcommand{\UpSub}[2]{\ensuremath{^{\text{#1}}_{\text{#2}}}}
\newcommand{\Spur}[1]{t\Sub{#1}}
\newcommand{\Ti}{\Spur{1}}
\newcommand{\Tii}{\Spur{2}}
\newcommand{\Tiii}{\Spur{3}}
\newcommand{\Tiv}{\Spur{4}}
\newcommand{\Ck}{\CheckmarkBold}
\newcommand{\Fl}{\XSolidBrush}
\newcommand{\xxx}{\hspaceThis{[}}
\newcommand{\zB}{z.\,B.\ }
\newcommand{\down}[1]{\ensuremath{\mathrm{#1}}}
\newcommand{\Zeile}{\vspace{\baselineskip}}
\newcommand{\Halbzeile}{\vspace{0.5\baselineskip}}
\newcommand{\Viertelzeile}{\vspace{0.25\baselineskip}}
\newcommand{\KTArr}[1]{\ding{222}~\fbox{#1}~\ding{222}}
\newcommand{\Ast}{*}
\newcommand{\SL}{\ensuremath{\llbracket}}
\newcommand{\SR}{\ensuremath{\rrbracket}}
\def\lspbottomrule{\bottomrule}
\def\lsptoprule{\toprule}
\newcommand{\Sw}[1]{\begin{sideways}#1\end{sideways}}
\newcommand{\Lab}{\ensuremath{\langle}}
\newcommand{\Rab}{\ensuremath{\rangle}}
\newcommand{\AbUmlautBreaker}{}
\ifdefined\HANDOUT
  \renewcommand{\AbUmlautBreaker}{\ /}
\fi
\newcommand{\LocStrutGrph}{\hspace{0.1\textwidth}}
\newcommand{\Nono}{---}

\newcommand{\Bewegtes}[1]{\ensuremath{_{\textrm{#1}}}}
\newcommand{\ORi}{\Bewegtes{1}}
\newcommand{\ORii}{\Bewegtes{2}}
\newcommand{\ORiii}{\Bewegtes{3}}
\newcommand{\ORiv}{\Bewegtes{4}}
\newcommand{\ORv}{\Bewegtes{5}}


% Beamer ==============================================================

\usetheme[hideothersubsections]{Boadilla}

\ifdefined\HANDOUT
  \usecolortheme{whale}
\else
  \usecolortheme{magpie}
\fi

\renewcommand<>{\rot}[1]{%
  \alt#2{\beameroriginal{\rot}{#1}}{#1}%
}
\renewcommand<>{\blau}[1]{%
  \alt#2{\beameroriginal{\blau}{#1}}{#1}%
}
\renewcommand<>{\orongsch}[1]{%
  \alt#2{\beameroriginal{\orongsch}{#1}}{#1}%
}
\renewcommand<>{\gruen}[1]{%
  \alt#2{\beameroriginal{\gruen}{#1}}{#1}%
}

\setbeamercolor{alerted text}{fg=trueblue}

\newcounter{lastpagemainpart}

\resetcounteronoverlays{exx}

\AtBeginSection[]{
  \begingroup
  \setbeamertemplate{navigation symbols}{}
  \begin{frame}[noframenumbering,plain]
  \vfill
  \centering
  \begin{beamercolorbox}[sep=8pt,center,shadow=true,rounded=true]{title}
    \usebeamerfont{title}\insertsectionhead\par%
  \end{beamercolorbox}
  \vfill
  \end{frame}
  \endgroup
}

\setbeamertemplate{itemize item}[circle]
\setbeamertemplate{enumerate item}[square]

\setbeamertemplate{navigation symbols}{\insertframenumber/\inserttotalframenumber\hspace{5em}}

% Tikz ================================================================

\usetikzlibrary{positioning,arrows,cd}
\tikzset{>=latex}

% Forest

\forestset{
  Ephr/.style={draw, ellipse, thick, inner sep=2pt},
  Eobl/.style={draw, rounded corners, inner sep=5pt},
  Eopt/.style={draw, rounded corners, densely dashed, inner sep=5pt},
  Erec/.style={draw, rounded corners, double, inner sep=5pt},
  Eoptrec/.style={draw, rounded corners, densely dashed, double, inner sep=5pt},
  Ehd/.style={rounded corners, fill=gray, inner sep=5pt,
    delay={content=\whyte{##1}}
  },
  Emult/.style={for children={no edge}, for tree={l sep=0pt}},
  phrasenschema/.style={for tree={l sep=2em, s sep=2em}},
  decide/.style={draw, chamfered rectangle, inner sep=2pt},
  finall/.style={rounded corners, fill=gray, text=white},
  intrme/.style={draw, rounded corners},
  yes/.style={edge label={node[near end, above, sloped, font=\scriptsize]{Ja}}},
  no/.style={edge label={node[near end, above, sloped, font=\scriptsize]{Nein}}},
  sake/.style={tier=preterminal},
  ake/.style={
    tier=preterminal
    },
}

\tikzset{
    invisible/.style={opacity=0,text opacity=0},
    visible on/.style={alt=#1{}{invisible}},
    alt/.code args={<#1>#2#3}{%
      \alt<#1>{\pgfkeysalso{#2}}{\pgfkeysalso{#3}}
    },
}

\forestset{
  visible on/.style={
    for tree={
      /tikz/visible on={#1},
      edge+={/tikz/visible on={#1}}}}}

\useforestlibrary{edges}

\forestset{
  narroof/.style={roof, inner xsep=-0.25em, rounded corners},
  forky/.style={forked edge, fork sep-=7.5pt},
  bluetree/.style={for tree={trueblue}, for children={edge=trueblue}},
  orongschtree/.style={for tree={orongsch}, for children={edge=orongsch}},
  rottree/.style={for tree={rot}, for children={edge=rot}},
  gruentree/.style={for tree={gruen}, for children={edge=gruen}},
  tuerkistree/.style={for tree={tuerkis}, for children={edge=tuerkis}},
  brauntree/.style={for tree={braun}, for children={edge=braun}},
  grautree/.style={for tree={grau}, for children={edge=grau}}, 
  gruennode/.style={gruen, edge=gruen},
  graunode/.style={grau, edge=grau},
}


% Drawing sonority diagrams =========================================== 

\makeatletter

\long\def\ifnodedefined#1#2#3{%
  \@ifundefined{pgf@sh@ns@#1}{#3}{#2}}

\newcommand\aeundefinenode[1]{%
  \expandafter\ifx\csname pgf@sh@ns@#1\endcsname\relax
  \else
    \typeout{Undefining node "#1"}%
    \global\expandafter\let\csname pgf@sh@ns@#1\endcsname\relax
  \fi
}

\newcommand\aeundefinethesenodes[1]{%
  \foreach \myn  in {#1}
    {%
      \ifnodedefined{\myn}{%
      \expandafter\aeundefinenode\expandafter{\myn}%
    }{}
    }%
}

\newcommand\aeundefinenumericnodes{%
  \foreach \myn in {1,2,...,50}
    {%
      \ifnodedefined{\myn}{%
      \expandafter\aeundefinenode\expandafter{\myn}%
    }{}
    }%
}
\makeatother

\newcommand{\plo}{0}
\newcommand{\fri}{0.5}
\newcommand{\nas}{1}
\newcommand{\liq}{1.5}
\newcommand{\vok}{2}

% Save text.
\newcommand{\lastsaved}{}
\newcommand{\textsave}[1]{\gdef\lastsaved{#1}#1}

\newcommand{\SonDiag}[2][0]{%
  \begin{tikzpicture}
    \textsave{.}
    \tikzset{
      normalseg/.style={fill=white},
      extrasyll/.style={circle, draw, fill=white},
      sylljoint/.style={diamond, draw, fill=white}
    }
    \node at (0,\plo) {P};
    \node at (0,\fri) {F};
    \node at (0,\nas) {N};
    \node at (0,\liq) {L};
    \node at (0,\vok) {V};

    % Draw the helper lines if required.
    \ifthenelse{\equal{#1}{0}}{}{%
      \foreach \y in {\plo, \fri, \nas, \liq,\vok} {%
	\draw [dotted, |-|] (0.25, \y) -- (#1.75, \y);
      }
    }

    \foreach [count=\x from 1, remember=\x as \lastx] \p / \y / \g in #2 {
      \ifthenelse{\equal{\y}{-1}}{\textsave{.}}{%

	% Draw the node, either plain, as Silbenbgelenk, or as extrasyllabic.
        \ifthenelse{\equal{\g}{1}}{%
	  \node (\x) [sylljoint] at (\x, \y) {\p};
	}{%
	  \ifthenelse{\equal{\g}{2}}{%
	    \node (\x) [extrasyll] at (\x, \y) {\p};
	  }{%
	    \node (\x) [normalseg] at (\x, \y) {\p};
	  }
	}

	% Draw the connection unless the previous node was not or was empty.
	\ifthenelse{\NOT\equal{\lastsaved}{.}}{%
	  \draw [->] (\lastx) to (\x);
	}{}
	\textsave{1}
      }
    }
    \aeundefinenumericnodes
  \end{tikzpicture}
}


\setbeamertemplate{navigation symbols}{}
\setbeamertemplate{section in toc}[circle]
\setbeamertemplate{subsection in toc}[square]
\setbeamertemplate{subsubsection in toc}[triangle]

\setbeamerfont{section in toc}{size=\tiny}
\setbeamerfont{subsection in toc}{size=\tiny}

