\section{Freiheitsgrade}

\begin{frame}
  {Freiheitsgrade "`intuitiv"'}
  \begin{itemize}[<+->]
    \item Beispiel: Schätzung eines Parameters (\zB Mittel)\\
      auf Basis von 1000 gemessenen Werten
    \item Wenn 999 Werte bekannt sind,\\
      steht abhängig vom Mittel der 1000ste Wert fest.
      \Zeile
    \item Für jedes Mittel $\mu$ einer Stichprobe mit $n$ Messungen\\
      sind also nur $n-1$ frei wählbar.
  \end{itemize}
\end{frame}

\begin{frame}
  {(Unintuitive) Erweiterung(en)}
  \begin{itemize}[<+->]
    \item generell: \alert{$df=n-|E|$}\\
      wobei $E$ die zu schätzenden Parameter sind. $|E|$ ist ihre Anzahl.
    \item Warum bei \alert{$\chi^2$} dann \alert{$df=(Zeilenzahl-1)\cdot(Spaltenzahl-1)$}?
      \Zeile
    \item Bsp.: Tabelle mit $2\times3$ Feldern, also $df=(2-1)(3-1)=1\cdot2=2$\ldots
    \item Bei bekannten Randsummen sind aber tatsächlich nur 2 Felder frei wählbar!
  \end{itemize}
  \begin{center}
    \visible<4->{
      \begin{tabular}[h!]{|c|c|c|c}
	\cline{2-3}
	\multicolumn{1}{c|}{}& X1 & X2 \\
	\cline{1-3}
	Y1 & $\oplus$ & & ZS1 \\
	\cline{1-3}
	Y2 & $\oplus$ & & ZS2 \\
	\cline{1-3}
	Y3 & & & ZS3 \\
	\cline{1-3}
	\multicolumn{1}{c}{}& \multicolumn{1}{c}{SQ1} & \multicolumn{1}{c}{SQ2} & \\
      \end{tabular}
    }
  \end{center}
\end{frame}


\section{Mehr zu Zähldatentests}


\subsection[Effektstärke]{Effektstärke für $\chi^2$: Cramérs $v$ und $\phi$}

\begin{frame}{Effektstärke}

  Der $\chi^2$-Wert sagt nichts über die \alert{Stärke eines Zusammenhangs}!\\
  \onslide<2->{Bei höheren absoluten Frequenzen wird auch der $\chi^2$-Wert größer.}

  \begin{figure}[h]
    \centering
    \begin{tabular}{|c|c|c|}
      \hline
      &  haben & sein\\
      \hline
      nord   &  27      & 33 \\
      \hline
	sued   &   3      & 34 \\
      \hline
    \end{tabular}~$\chi^2$ = \onslide<4->{\alert{12,89}}~\visible<6->{
    \begin{tabular}{|c|c|c|}
      \hline
	    &  haben & sein\\
      \hline
	nord   &  27.84\% & 34.02\% \\
      \hline
	sued   &  3.09\%     & 35.05\% \\
      \hline
    \end{tabular}
    }
  \end{figure}

  \begin{figure}[h]
    \centering
    \begin{tabular}{|c|c|c|}
      \hline
	    &  haben & sein\\
      \hline
	nord   &  54      & 66 \\
      \hline
	sued   &  6     & 68 \\
      \hline
      \end{tabular}~$\chi^2$ = \onslide<5->{\alert{27,46}}~\visible<7->{
      \begin{tabular}{|c|c|c|}
      \hline
	    &  haben & sein\\
      \hline
	nord   &  27.84\%      & 34.02\% \\
      \hline
	sued   &   3.09\%      & 35.05\% \\
      \hline
    \end{tabular}
  }
  \end{figure}
\end{frame}


\begin{frame}{Effektstärke II}
  Pearsons $\phi$: Maß für die Stärke des Zusammenhangs in 2$\times$2-Tabellen
    
  \begin{figure}
    \centering
    \alert{$\phi = \sqrt{\frac{\chi^2}{n}}$}
  \end{figure}

  \visible<2->{$\phi$ ist eine Zahl zwischen 0 und 1:\\
  Je größer, desto stärker der Zusammenhang zwischen den Variablen.}

  \visible<3->{
  \begin{figure}
  \centering
  Beispiel: \(\phi = \sqrt{\frac{\chi^2}{n}} = \sqrt{\frac{12.89}{97}} = \onslide<1->{\alert{0.3648}\)}}
  \end{figure}
\end{frame}


\begin{frame}
  {Cramérs $v$}

  Cramérs $v$ für $n\times n$-Tabellen mit $n>2$ oder $m>2$

  \begin{center}
    \alert{$v=\sqrt{\frac{\frac{\chi^2}{n}}{min(s-1,z-1)}}$}\\
    mit: $s$ die Spaltenzahl und $z$ die Zeilenzahl
  \end{center}

  \vspace{1cm}
  \footnotesize
  Beachte: für $2\times2$-Tabellen: $s-1=1$ und $z-1=1$,\\[2ex]
  also $min(s-1,z-1)=1$\\[1ex]
  daher: $v=\sqrt{\frac{\ \ \frac{\chi^2}{n}\ \ }{1}}=\sqrt{\frac{\chi^2}{n}}=\phi$

\end{frame}


\begin{frame}
  {In \texttt{R}}
  Speichern des Test-Objekts:\\
  \texttt{> my.chi2.test <- chisq.test(my.matrix)}\\
  \vspace{0.5cm}
  
  Speichern des $\chi^2$-Werts mit:\\
  \texttt{> my.chi2.value <- as.numeric(my.chi2.test\$statistic})\\
  \vspace{0.5cm}
  
  Speichern von $n$:\\
  \texttt{> my.n <- sum(my.matrix)}\\
  \vspace{0.5cm}

  Also Effektstärke (mit Ausgabe):\\
  \texttt{> my.phi <- sqrt( my.chi2.value / my.n ); my.phi}
\end{frame}


\subsection[Chancenverhältnis]{Chancenverhältnis}

\begin{frame}
  {Chance (odds)}
  \begin{itemize}[<+->]
    \item Die \alert{Chance (odds)} $o$ setzt die Wahrscheinlichkeit $p$ eines Ereignisses $E$\\
      in Relation zur Gegenwahrscheinlichkeit:
  \end{itemize}
  \pause
   \begin{center}
     \alert{$o(E)=\frac{p(E)}{1-p(E)}$}\\[2ex]
     \pause
     und damit\\[2ex]
     \pause
     \alert{$p(E)=\frac{o(E)}{1+o(E)}$}
  \end{center}
  \pause
  \begin{itemize}[<+->]
    \item Ein Ereignis ist in Korpusstudien i.\,d.\,R.\\\
      das Auftreten einer \alert{Variablenausprägung}.
    \item Die Information in den Maßen Wahrscheinlichkeit und Chance\\
      ist dieselbe (s. Umrechenbarkeit ineinander).
  \end{itemize}
\end{frame}

\begin{frame}
  {Chance und Wahrscheinlichkeit und Zähldaten}
  \vspace{-1cm}
      \begin{center}
	\begin{tabular}{|c|c|}
	      \hline
	      \textbf{Aux} & \textbf{Anzahl} \\
	      \hline
	      haben   &  27   \\
	      \hline
	      sein   &  33   \\
	      \hline
	    \end{tabular}\\
      \end{center}
      \pause
      $p(haben)=\frac{27}{27+33}=\frac{27}{60}=0.45$ (Wahrscheinlichkeit)\\[2ex]
      \pause
      $1-p(haben)=p(\neg haben)=\frac{33}{27+33}=\frac{33}{60}=0.55$ (\alert{Gegenwahrscheinlichkeit})\\[2ex]
      \pause
      Beachte: $p(haben)+p(\neg haben)=1$\\[2ex]
      \pause
      \alert{$o(haben)=\frac{\ \frac{27}{60}\ }{\frac{33}{60}}=\frac{27}{60}\cdot\frac{60}{33}=\frac{27}{33}=0.82$}\\[2ex]
      \pause
      allgmein: \alert{$p(E)=\frac{Anzahl(E)}{Anzahl(E)+Anzahl(\neg E)}$} und \alert{$o(E)=\frac{Anzahl(E)}{Anzahl(\neg E)}$}\\[2ex]
\end{frame}

\begin{frame}
  {Chancenverhältnis (odds ratio)}
  \begin{itemize}
    \item Das \alert{Chancenverhältnis (odds ratio)} gibt das Verhältnis an, wie sich\\
      die Chancen einer Variablenausprägung $E$ unter Bedingung $A$ --\\
      also $o(E|A)$ -- und unter Bedingung $B$ -- also $o(E|B)$ -- \\
      zueinander Verhalten:
  \end{itemize}
  \begin{center}
    \alert{$r(E|A, E|B)=\frac{o(E|A)}{o(E|B)}$}
  \end{center}
\end{frame}

\begin{frame}
  {Beispiel zum Chancenverhältnis (1)}
  \begin{itemize}
    \item Wir haben Texte aus Süddeutschland und Norddeutschland auf\\
      das Auftreten des Perfektauxiliars \textit{haben} und \textit{sein} bei\\
      bestimmten Verben untersucht.
    \item Die Kreuztabelle:
      \begin{center}
	\begin{tabular}{|c|c|c|}
	      \hline
	      &  nord & sued \\
	      \hline
	      haben   &  27      & 3   \\
	      \hline
	      sein   &  33      & 34  \\
	      \hline
	\end{tabular}
      \end{center}
  \end{itemize}
\end{frame}

\begin{frame}
  {Beispiel zum Chancenverhältnis (2)}
    \begin{center}
      \scalebox{0.7}{
	\begin{tabular}{|c|c|c|}
	      \hline
	      &  nord & sued \\
	      \hline
	      haben   &  27      & 3   \\
	      \hline
	      sein   &  33      & 34  \\
	      \hline
	\end{tabular}
      }
    \end{center}
    \begin{itemize}
      \item $o(haben|nord)=\frac{27}{33}=0.82$
      \item $o(haben|sued)=\frac{3}{34}=0.09$
	\pause\pause\pause
      \item Verhältnis zwischen den Chancen: \onslide<5->{$or=\frac{0.82}{0.09} = 9.11$}
	\pause\pause
      \item D.\,h.\ die Chance von \textit{haben} ist 9.11 mal größer, wenn \textit{Region} \textit{nord} ist.
	\pause
      \item Ersatz für Effektstärke bei Fisher-Test
    \end{itemize}
\end{frame}

\subsection{Binomialtest}

\begin{frame}
  {Bernoulli-Experimente}
  \begin{itemize}[<+->]
    \item binäre Daten: Ereignis vs.\ Nicht-Ereignis bzw.\ Ja\slash Nein
      \vspace{0.5cm}
    \item Vgl. Behauptung: "`Gen/Dat alternieren frei bei \textit{wegen}."'
      \begin{itemize}[<+->]
	\item "`frei alternieren"' = beide Kasus haben die gleiche Chance.
	\item Grundgesamtheit per Hypothese: \alert{50\% Genitive} und \alert{50\% Dative}
      \end{itemize}
      \vspace{0.5cm}
    \item Korpusstichprobe: \alert{F(Genitiv)=41} und \alert{F(Dativ)=59}
    \item Passt das zur Hypothese bei sig=0.05?
  \end{itemize}
\end{frame}

\begin{frame}
  {Binomialtest}
  \begin{itemize}[<+->]
    \item H0: Es gibt keine Abweichung von der erwarteten Wahrscheinlichkeit.
      \Zeile
    \item \alert{H0: $p(Dativ)=0.5$}
  \end{itemize}
\end{frame}

\begin{frame}
  {Binomialtest im Einzelnen}
  Benötigte Größen:

  \begin{itemize}[<+->]
    \item Stichproben der Größe \alert{$n$}
    \item H0-Wahrscheinlichkeit \alert{$p$} (hier $p=0.5$)
    \item Anzahl der beobachteten Ereignisse: \alert{X} (hier $X(Dativ)=59$)
  \end{itemize}
\end{frame}

\begin{frame}
  {Unter Annahme der H0\ldots}
  \begin{itemize}[<+->]
    \item Wenn \alert{$p\cdot n>10$ und $(1-p)\cdot n>10$}\\
      approximiert die Binomialverteilung die Normalverteilung.
      \Zeile
    \item Es gilt dann (unter Annahme der H0!) für die Normalverteilung:
      \Zeile
      \begin{itemize}
	\item Mittel: \alert{$\mu=p\cdot n$}
	\item Standardabweichung: \alert{$s=\sqrt{n\cdot p\cdot(1-p)}$}
	\item Wir können für den gemessenen Wert den z-Wert ausrechnen.
      \end{itemize}
  \end{itemize}
  \pause
  \vspace{0.5cm}
  \begin{center}
    \alert{$z=\frac{X-\mu}{s}=\frac{X-p\cdot n}{\sqrt{n\cdot p\cdot (1-p)}}$}
  \end{center}
\end{frame}

\begin{frame}
  {Ausrechnen des Beispiels und Signifikanz}
  \begin{center}
    $z=\frac{59-(0.5\cdot 100)}{\sqrt{100\cdot 0.5\cdot 0.5}}=\frac{59-50}{\sqrt{25}}=\frac{9}{5}=1.8$
  \end{center}
  \pause
  \begin{itemize}[<+->]
    \item Der gemessene Wert liegt 1.8 Standardabweichungen\\
      vom H0-Mittel entfernt.
    \item Wir kennen bereits die kritischen Werte für Normalverteilungen\\
      und sig=0.05: \alert{$-1.96 .. 1.96$}
    \item Die H0 kann also nicht zurückgewiesen werden bei sig=0.05.
      \Zeile
    \item \alert{Interpretation: Wir haben keine Evidenz dafür,\\
      dass die Variation in der Grundgesamtheit von\\
    einer 50:50-Verteilung abweicht.}
      \Zeile
    \item \rot{Falsche Interpretation: Wir haben Evidenz dafür,\\
      dass die Verteilung in der Grundgesamtheit 50:50 ist.}
  \end{itemize}
\end{frame}

\begin{frame}
  {In \texttt{R}}
  \begin{center}
    \texttt{> binom.test(59, 100, 0.5)}
  \end{center}
\tt\footnotesize
\ \ \ \ \ Exact binomial test\\[4ex]

data:  59 and 100\\
number of successes = 59, number of trials = 100, p-value = 0.08863\\
alternative hypothesis: true probability of success is not equal to 0.5\\
95 percent confidence interval:\\
\ \ 0.4871442\ \ 0.6873800
sample estimates:\\
probability of success 0.59 \\
\end{frame}

\section{Effektstärken bei t-Test und ANOVA}

\subsection{Ein-Stichpropben-t-Test}

\begin{frame}
  {Effektstärke Ein-Stichproben-t-Test}
  \begin{itemize}[<+->]
    \item Signifikanz $\neq$ starker Effekt
    \item Effektstärke beim t-Test für Stichprobe $x$:
  \end{itemize}
  \pause
  \begin{center}
    \alert{Cohens $d=\frac{\bar{x}-\mu}{s(x)}$}
  \end{center}
  \pause
  \begin{itemize}
    \item Herleitung\slash Erklärung: Gravetter \& Wallnau, Kap.\ 9
  \end{itemize}
\end{frame}

\begin{frame}
  {Erklärung der Varianz}
  \begin{itemize}
    \item ähnlich der Effektstärke: \\
      \alert{Welcher Anteil der Varianz in den Daten\\
      wird durch die Unabhängige erklärt?}
  \end{itemize}
  \pause
  \begin{center}
    \alert{Cohens $r^2=\frac{t^2}{t^2+df}$}
  \end{center}
  \pause
  \begin{itemize}
    \item Herleitung\slash Erklärung: Gravetter \& Wallnau, Kap.\ 9
  \end{itemize}
\end{frame}

\subsection{Zwei-Stichproben-t-Test}



\begin{frame}
  {Effektstärke Zwei-Stichproben-t-Test}
  Effektstärke
  \begin{center}
    $d=\frac{\bar{x_1}-\bar{x_2}}{\sqrt{s^2_p}}$
  \end{center}
  \pause
  Erklärung der Varianz
  \begin{center}
    $r^2=\frac{t^2}{t^2+df}$
  \end{center}
\end{frame}


\subsection{ANOVA}

\begin{frame}
  {Effektstärke einfaktorielle ANOVA}

  \begin{center}
    \alert{$\eta^2=\frac{SQ_{zwischen}}{SQ_{gesamt}}$}\\[4ex]
    (wieder ein $r^2$-Maß)
  \end{center}
\end{frame}



\begin{frame}
  {Effektstärken bei der zweifaktoriellen ANOVA}
  Entsprechend sind \alert{drei} $\eta^2$ auszurechnen:\\

  \vspace{0.5cm}
  \begin{center}
    $\eta^2_A=\frac{SQ_A}{SQ_{gesamt} - SQ_B - SQ_{A\times B}}$ \\[3ex]
    $\eta^2_B=\frac{SQ_B}{SQ_{gesamt} - SQ_A - SQ_{A\times B}}$ \\[3ex]
    $\eta^2_{A\times B}=\frac{SQ_{A\times B}}{SQ_{gesamt} - SQ_A - SQ_B}$ \\
  \end{center}

  Wir fragen jeweils, welchen Anteil an der Varianz,\\
  die die anderen beiden Faktoren \alert{nicht} erklären,\\
  der jeweilige dritte Faktor hat.
\end{frame}



\section{Voraussetzungen für t-Test und ANOVA}

\begin{frame}
  {Caveat}
  \Zeile
  \begin{center}
    Bedingung für \rot{alle} Tests:\\
    \alert{Unabhängigkeit} der Messungen \\
    \Zeile
    Wenn bei t-Test oder ANOVA also gepaarte Stichproben vorliegen\\
    (Messung derselben Proband*innen unter Bedingung 1 und 2 usw.):\\
    \alert{Besondere Versionen für geparte Stichproben nehmen!}\\
    \Zeile
    Details hier nicht besprochen.
  \end{center}
\end{frame}


\begin{frame}
  {Voraussetzungen prüfen I}
  Die \alert{GGs müssen normalerverteilt} sein:
  \begin{center}
    \texttt{shapiro.test(x)}\\
    Wenn $p\leq0.05$ wird die Nullhypothese des Shapiro-Wilk-Tests verworfen.\\
    H0: Die Werte stammen aus einer normalverteilten GG.
  \end{center}
  \Zeile
  \pause
  Die \alert{Varianzen müssen homogen sein}:
  \begin{center}
    \texttt{var.test(x1, x2)}\\
    Auch hier: $p\leq0.05$ weist die H0 zurück.\\
    H0: Die Varianzen von x1 und x2 sind homogen.
  \end{center}
\pause
  \Halbzeile
\begin{center}
  \alert{Solche Tests sind umstritten, weil sie angeblich zu empfindlich reagieren.\\
    \cite{ZuurEa2009} empfehlen \zB grafische Methoden. Ich nicht.}
\end{center}

\end{frame}


\begin{frame}
  {Voraussetzungen prüfen II}
  Wenn Voraussetzungen nicht erfüllt sind:
  \begin{itemize}[<+->]
    \item steigt das Risiko für Typ 1-Fehler
      \Zeile
    \item nicht-parametrische Alternative nehmen
    \item Daten transformieren (Logarithmus für Normalverteilung)
      \Zeile
    \item sich über Robustheit des Test ggü. verletzten Annahmen informieren\\
      (oft schwer zugängliche und kontroverse Spezialliteratur)
  \end{itemize}
\end{frame}


\section[Alternativen]{Nichtparametrische Alternativen zu t-Test und ANOVA}

\begin{frame}
  {Übersicht}
  \begin{itemize}[<+->]
    \item Alternativen, wenn Bedingungen für t-Test und ANOVA \\
      nicht erfüllt sind (Normalverteilung, Varianzhomogenität)
      \Zeile
    \item Prinzip: \alert{Umrechnen von Werten in Ränge}
    \item nicht-parametrische Tests
  \end{itemize}
\end{frame}

\begin{frame}
  {Literatur}
  \begin{itemize}
    \item \cite{BortzLienert2008}
    \item \cite{GravetterWallnau2007}
  \end{itemize}
\end{frame}

\begin{frame}
  {Übersicht}
  \begin{itemize}[<+->]
    \item Mann-Whitney U-Test: Alternative zum t-Test mit zwei Stichproben
    \item Kruskal-Wallis H-Test: Alternative zur einfaktoriellen ANOVA
  \end{itemize}
\end{frame}

\subsection{Mann-Whitney U-Test}

\begin{frame}
  {Wiederholung: Bedingungen für t-Test}
  \begin{itemize}[<+->]
    \item Intervallskalierung der Abhängigen
    \item Normalität der Abhängigen
    \item Varianzhomogenität der Abhängigen in den Gruppen
    \item Unabhängigkeit der Messungen
      \Zeile
  \end{itemize}
  \pause
  \begin{center}
    Alle bis auf die letzte entfallen beim Mann-Whitney U-Test.
  \end{center}
\end{frame}

\begin{frame}
  {Direkte Berechnung beim MWU}
  Gruppen\slash Stichproben (Messwerte):\\
  $x_1=[9,8,12,16]$\\
  $x_2=[4,11,7,13]$\\[2ex]
  \pause
  Ränge in der \alert{zusammengelegten} Stichprobe:\\
  $X=[4,7,8,9,11,12,13,16]$\\
  $R(x_1)=[4,3,6,8]$\\
  $R(x_2)=[1,5,2,7]$\\[2ex]
  \pause
  Addiere für jeden Wert beider Gruppen die Anzahl der\\
  \alert{niedrigeren Ränge (=höhere Rangzahl!)} in der anderen Gruppe:\\
  $U(x_1)=2+2+1+0=5$\\
  $U(x_2)=4+2+4+1=11$\\
  \alert{$U=min(U_{x_1}, U_{x_2})=U_{x_1}=5$}
\end{frame}

\begin{frame}
  {Allgemeine Formel}
  \begin{center}
    \alert{$U(x_{\alpha})=n_1\cdot n_2+\frac{n_{\alpha}(n_{\alpha}+1)}{2}-\sum R(x_{\alpha})$}
  \end{center}
  \pause
  \begin{itemize}[<+->]
    \item $\sum R(x_1)=4+3+6+8=21$
    \item $\sum R(x_2)=1+5+2+7=15$
    \item $n_1\cdot n_2=4\cdot 4=16$
    \item $n_1(n_1+1)=n_2(n_2+1)=4\cdot5=20$
    \item $U(x_1)=16+10-21=5$
    \item $U(x_2)=16+10-15=11$
    \item \alert{$U=5$}
  \end{itemize}
\end{frame}

\begin{frame}
  {Siginifikanz und Effektstärke}
  \begin{itemize}[<+->]
    \item Signifikanz für kleine Stichproben: \alert{Tabelle}
    \item bei großen Stichproben: U ugf.\ normalverteilt, also \alert{z-Test}
    \item in \texttt{R}:
  \end{itemize}
  \pause
  \begin{center}
    \alert{\texttt{> wilcox.test(x1,x2, paired = FALSE)}}
  \end{center}
  \pause
  \Zeile
  \begin{itemize}[<+->]
    \item Effektstärke: Punkt-biserielle Korrelation
    \item entspricht Pearson-Korrelation, aber Unabhängige ist dichotom
    \item In \texttt{R}: \alert{\texttt{cor(c(x1,x2), c(rep(0,4),rep(1,4)))}}
  \Zeile
    \item alternativ: "`relativer Effekt"' (Bortz \& Lienert, S.\ 142)
  \end{itemize}
\end{frame}

\begin{frame}
  {Probleme}
  \begin{itemize}[<+->]
    \item Bei sehr vielen gleichen Rängen ist\\
      der Mann-Whitney U-Test unzuverlässig.
    \item Bei gleichen Rängen generell: korrigierte Version\\
      (s. Bortz \& Lienert, S.\ 146).
      \Zeile
    \item Er ist daher nur begrenzt geeignet für Dinge wie 5-Punkt-Skalen.
    \item generell am stärksten bei gleich großen und\\
      gleich stark streuenden Stichproben
      \Zeile
    \item letzter Ausweg: \alert{Mediantest} (Bortz \& Lienert, S.\ 137)
  \end{itemize}
\end{frame}

\subsection{Kruskal-Wallis H-Test}

\begin{frame}
  {Mehr als zwei Gruppen}
  Wie vom t-Test zur ANOVA\ldots\\
  $x_1=[9,8,12,16]$\\
  $x_2=[4,11,7,13]$\\
  $x_3=[13,12,5,15]$\\[2ex]
  \pause
  Gleiches Vorgehen wie bei Mann-Whitney über\\
  \alert{Rang in der zusammengelegten Stichprobe}:\\
  \begin{center}
    \begin{tabular}[h!]{|c||c|c|c|c|c|c|c|c|c|c|c|c|}
      \hline
      X    & 4 & 5 & 7 & 8 & 9 & 11 & 12 & 12 & 13 & 13 & 15 & 16 \\
      \hline
      R(X) & 1 & 2 & 3 & 4 & 5 & 6 & \multicolumn{2}{c|}{7.5} & \multicolumn{2}{c|}{9.5} & 11 & 12 \\
      \hline
    \end{tabular}
  \end{center}
  \pause
  $R(x_1)=[5,4,7.5,12]$\\
  $R(x_2)=[1,6,3,9.5]$\\
  $R(x_3)=[9.5,7.5,2,11]$\\
\end{frame}

\begin{frame}
  {Berechnung des Kruskal-Wallis H-Werts}
  \begin{center}
    \alert{$H=\frac{12}{N(N+1)}\cdot\sum\limits_i \frac{(\sum R(x_i))^2}{n_i}-3(N+1)$}
  \end{center}
  \pause
  Am Beispiel:
  \begin{itemize}[<+->]
    \item Gruppen-Rang-Summen:
      \begin{itemize}
	\item $R(x_1)=[5,4,7.5,12]$, \alert{$\sum R(x_1)=28.5$}
	\item $R(x_2)=[1,6,3,9.5]$, \alert{$\sum R(x_2)=19.5$}
	\item $R(x_3)=[9.5,7.5,2,11]$, \alert{$\sum R(x_3)=30$}
      \end{itemize}
    \item $\alert{H}=\frac{12}{12\cdot(12+1)}\cdot(\frac{28.5^2}{4}+\frac{19.5^2}{4}+\frac{30^2}{4})-3(12+1)=$
    \item $0.077\cdot(203.06+95.06+225)-39=\alert{1.28}$
  \end{itemize}
\end{frame}

\begin{frame}
  {Signifikanztest}
  \begin{itemize}[<+->]
    \item Bei $n>5$ ist H unter der H0 \alert{$\chi^2$-verteilt}.
    \item mit \alert{$df=k-1$} ($k$ ist die Anzahl der Gruppen)
      \Zeile
    \item Effektstärke: tja\ldots
    \item "`relative Effekte"' sind rechenbar (Bortz \& Lienert, S.\ 159)
  \end{itemize}
\end{frame}

\begin{frame}
  {In \texttt{R}}
  \begin{center}
    \alert{\texttt{> kruskal.test(c(x1,x2,x3)~c(rep(0,4),rep(1,4),rep(2,4)))}}
  \end{center}
  Rechnen Sie bitte mal die U- und H-Tests von diese Folien\\
  und vergleichen Sie die p-Werte mit denen von t-Test und ANOVA\\
  über die gleichen Daten:\\[3ex]
  $x_1=[9,8,12,16]$\\
  $x_2=[4,11,7,13]$\\
  $x_3=[13,12,5,15]$
\end{frame}

\ifdefined\TITLE
  \section{Nächste Woche | Überblick}

  \begin{frame}
    {Einzelthemen}
    \begin{enumerate}
      \item Inferenz
      \item Deskriptive Statistik
      \item Nichtparametrische Verfahren
      \item z-Test und t-Test
      \item ANOVA
      \item Freiheitsgrade und Effektstärken
      \item \alert{Power und Severity}
      \item Lineare Modelle
      \item Generalisierte Lineare Modelle
      \item Gemischte Modelle
    \end{enumerate}
  \end{frame}
\fi
