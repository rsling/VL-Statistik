\section[Quantitativ]{Quantitative Analyse}

\begin{frame}
  {Übersicht}
  \begin{itemize}[<+->]
    \item Messungen und Experimente
    \item Hypothesen und Theorien
    \item Begriff der "`Variablen"'
    \item statistische Inferenz:
      \begin{itemize}[<+->]
	\item Hypothesenpaare H1 und H0
	\item Typ I- und Typ II-Fehler
	\item Validität
	\item exakte Hypothesentestung
      \end{itemize}
  \end{itemize}
\end{frame}

\begin{frame}
  {Literatur}
  \begin{itemize}
    \item \alert{\citet[Kap.\ 1 und 2]{MaxwellDelaney2004}}
    \item \citet[Kap.\ 1]{GravetterWallnau2007}
  \end{itemize}
\end{frame}

\subsection{Quantitative empirische Forschung}

\begin{frame}
  {Empirisch}
  \begin{itemize}[<+->]
    \item beobachtbare Phänomene
    \item Beobachtungen reproduzierbar
    \item messbar = beobachtbar (Sinneswahrnehmung an sich irrelevant)
    \item Realismus: "`wirkliche"' Phänomene und ihre Mechanismen
      \Zeile
    \item \alert{Experiment}
  \end{itemize}
\end{frame}

\begin{frame}
  {Empirie: Gründe für Reproduzierbarkeitsbedingung}
  \begin{itemize}[<+->]
    \item nicht-arbiträre Genauigkeit der Messung (Wirkung und Störeinflüsse)
    \item potentiell inadäquate Messung des theoretischen Konstrukts
    \item $\Rightarrow$ Vermeidung von Fehlschluss auf unechte Ursachen
    \item $\Rightarrow$ \alert{relevante Ursachen}
    \item Generalisierbarkeit der Ergebnisse
    \item Überkommen der individuell gefärbten Wahrnehmung
  \end{itemize}
\end{frame}

\begin{frame}
  {Individualgrammatik und fehlgeleitete Generative Grammatiker}
  \begin{itemize}[<+->]
    \item Gegenstand: interne (mentale) Grammatik (I-Grammatik)\\
      universeller und individueller Teil
    \item I-Grammatik bei jedem Sprecher (leicht) verschieden
    \item I-Grammatik erlaubt immer binäre Grammatikalitätsentscheidung
    \item also: \alert{Linguisten können eigene Grammatik untersuchen!}\\
      (Introspektion)
    \item universalistische Aussagen auf Basis solcher Ergebnisse zulässig
    \item \alert{Das ist auf allen Ebenen eine Frechheit!}
  \end{itemize}
\end{frame}

\begin{frame}
  {Hypothesen}
  \begin{itemize}[<+->]
    \item Ursprung der Hypothesen: Theorien
    \item interessante Hypothesen:
      \begin{itemize}[<+->]
	\item Formulierung relevanter Kausationsbedingung (wenn, dann)
	\item großer Gültigkeitsbereich (ein Sprecher vs.\ alle Sprecher)
      \end{itemize}
  \end{itemize}
\end{frame}

\begin{frame}
  {Hypothesenprüfung}
  \begin{itemize}[<+->]
    \item Kann die Hypothese weiter angenommen werden,\\
      oder liefert das Experiment starke Evidenz gegen sie?
    \item Probleme bei Prüfung:
      \begin{itemize}[<+->]
	\item störende Einflüsse
	\item ungenaue Operationalisierung (s.\,u.)
	\item stochastische Natur des Phänomens
	\item kleine Stichprobe (s.\,u.)
      \end{itemize}
    \item falsche Positive und falsche Negative jeweils zu vermeiden
  \end{itemize}
\end{frame}

\begin{frame}
  {Qualitative Forschung}
  \begin{itemize}[<+->]
    \item Daten besorgen\slash Experiment machen
    \item typischerweise \alert{kleine Datenmengen}
    \item Datenbetrachtung durch menschliche Wahrnehmung
    \item Interpretation der Daten durch Nachdenken
      \Zeile
    \item extrem fehleranfällig (Fehler = unzulässige Generalisierung)
    \item wichtig zur Hypothesengenerierung
  \end{itemize}
\end{frame}

\begin{frame}
  {Quantitative Forschung}
  \begin{itemize}[<+->]
    \item Daten besorgen\slash Experiment machen
    \item typischerweise \alert{größere Datenmengen}
    \item Datenanalyse durch Visualisierung\slash mathematische Datenbeschreibung
    \item Hypothesenprüfung durch \alert{Testverfahren} (s.\,u.)
    \item Grundlage moderner Wissenschaft
  \end{itemize}
\end{frame}

\begin{frame}
  {Operationalisierung, Theorie und Auxiliarhypothesen}
  \begin{itemize}[<+->]
    \item Operationalisierung = präzise Formulierung der Messmethode\\
      für ein theoretisches Konstrukt
    \item Bsp.\ Konstruk "`Satzlänge"': Wortanzahl? Phonemanzahl? Phrasenanzahl?
    \item Bsp.\ Konstrukt "`Satztopik"': Oha!?! \citep{CookBildhauer2013}
    \item alle genannten Beispiele: \alert{abhängig von Auxiliarhypothesen}\\
      bzw.\ anderen theoretischen Konstrukten (Wort, Phonem, Phrase, \dots)
  \end{itemize}
\end{frame}

\subsection{Grundbegriffe der quantitativen Forschung}

\begin{frame}
  {Grundgesamtheit}
  \begin{itemize}[<+->]
    \item von Interesse: allgemeine Gesetzmäßigkeiten
    \item also Untersuchungsgegenstand: \alert{alle X} (Sprecher, Sätze, \dots)
    \item untersuchbar: kleine Menge von X
    \item wenn "`alle X"' an sich sowieso sehr klein:\\
      interessante Fragestellung nur schwer möglich
      \Zeile
    \item fiktives Beispiel:
      \begin{itemize}[<+->]
	\item Sprecher ohne Ersatzinfinitiv (\textit{dass ich schlafen gemusst habe})\\
	  benutzen Modalverben öfter im Präteritum (\textit{dass ich schlafen musste})
	\item \dots aber es gibt nur einen Sprecher ohne Ersatzinfinitiv
	\item $\Rightarrow$ dann auch kein Theorie- und Empiriebedarf
      \end{itemize}
  \end{itemize}
\end{frame}

\begin{frame}
  {Stichprobe}
  \begin{itemize}[<+->]
    \item weil Grundgesamtheit intrinsisch nicht betrachtbar:\\
      \alert{Verwendung eingeschränkter Datensätze}
    \item ideal: \alert{uniform zufällige Stichprobe}\\
      = jedes Element der Grundgesamtheit hat die gleiche Chance beim Ziehen\\
    \item andere Möglichkeit: \alert{stratifizierte Stichprobe}\\
      = Stichprobe so zusammengesetzt, dass wichtige Eigenschaften\\
      proportional repräsentiert sind
    \item Problem bei Letzterem: haufenweise Auxiliarhypothesen
    \item außerdem: nicht unbedingt erforderlich, s.\ letzten Teil der heutigen VL
  \end{itemize}
\end{frame}

\begin{frame}
  {Induktive Inferenz}
  \begin{itemize}[<+->]
    \item Ziel: Schluss von Stichprobe auf Grundgesamtheit
    \item kritisch: Stichprobengröße
    \item Kriterium: Wahrscheinlichkeit, das Ergebnis per Zufall zu bekommen
    \item starker Einfluss: Stärke des Effekts in der Stichprobe
  \end{itemize}
\end{frame}

\begin{frame}
  {Variablen I}
  \begin{itemize}[<+->]
    \item meist uninteressanter Typ Fragestellung:\\
      "`Wieviel Prozent X haben Eigenschaft A?"'
    \item wegen \alert{Fehlens kausaler Zusammenhänge}
    \item Bsp.: wie oft \textit{wegen} mit Dat bzw.\ Gen?
    \item besser:\\
      \alert{"`Hängt bei X die Wahrscheinlichkeit für Eigenschaft B von A ab?"'}
    \item Bsp.: "`Nehmen denominale Präp eher Gen oder Dat?"'
  \end{itemize}
  \pause
  \begin{center}
    konzeptuell:
    \scalebox{0.7}{
      \begin{tabular}[h!]{|c||c|c|}
	\cline{2-3}
	\multicolumn{1}{c||}{} & denominale P & andere P \\
	\hline
	\hline
	Dat & $x_1$ & $x_2$\\
	\hline
	Gen & $x_3$ & $x_4$\\
	\hline
      \end{tabular}
    }
  \end{center}
\end{frame}

\begin{frame}
  {Variablen II}
  \begin{itemize}[<+->]
    \item Eigenschaften, quantitativ gemessen: \alert{Variablen}
    \item im Experiment:
      \begin{itemize}[<+->]
	\item \alert{kontrolliere} für Theorie irrelevante Variablen (\alert{Störvariablen})
	\item \alert{variiere} "`Ursachen-Variablen"' (\alert{unabhängige Variablen})
	\item \alert{beobachte} "`Wirkung-Variablen"' (\alert{abhängige Variablen})
      \end{itemize}
  \end{itemize}
\end{frame}

\begin{frame}
  {Experiment und Quasi-Experiment}
  \begin{itemize}[<+->]
    \item Problem in Astronomie, Korpuslinguistik usw.: keine Experimente möglich
    \item \alert{unabhängige Variablen nicht variierbar}
    \item Daten liegen bereits vor bzw.\ fallen vom Himmel
    \item Auswahl von Datensätzen, so dass von den unabhängigen Variablen\\
      die zur Theorieprüfung nötigen Permutationen im Datensatz vorkommen
    \item dabei Zusatzproblem bei Korpuslinguistik: Korpus meist nicht\\
      das eigene, wenig Informationen über mögliche Verzerrungen
      \Zeile
    \item \alert{Was ist die Grundgesamtheit?}
  \end{itemize}
\end{frame}

\begin{frame}
  {Hypothesenpaare}
  \begin{itemize}[<+->]
    \item \alert{H1 (eigene Hypothese)}:\\
      erwarteter Variablenzusammenhang aus Kausalrelation
    \item \alert{H0 (Nullhypothese)}: Negation der H1
    \item Ziel der Inferenzstatistik: \alert{Zurückweisung der H0}
    \item Logik: \textit{Entweder ein sehr seltenes Ereignis ist eingetreten,\\
      oder es besteht tatsächlich ein Zusammenhang.}
    \item $\Rightarrow$ Stärkung der H1, aber \alert{nicht "`Beweis"'}!
      \vspace{\baselineskip}
    \item geringe Bedeutung für sich genommen
    \item \alert{Severity} hängt von viel mehr Faktoren ab (Mayo)
    \item weitere Experimente\slash Replikationen
  \end{itemize}
\end{frame}

\begin{frame}
  {Mayos Strong Severity Principle}
      \vspace{\baselineskip}
  \begin{quote}
    \textit{Severity (strong): We have evidence for a claim C just to the extent it survives a stringent scrutiny.} If C passes a test that was highly capable of finding flaws or discrepancies from C, and yet none or few are found, then the passing result, \textbf{x}, is evidence for C. [Mayo 2018:14]
  \end{quote}
\end{frame}

\begin{frame}
  {Gerichtete Hypothesen}
  \begin{itemize}[<+->]
    \item übliche Hypothesen (H1):\\
      "`Es besteht ein Zusammenhang zwischen A und B."'
    \item H0 dann: Fehlen des Zusammenhangs
    \item = \alert{ungerichtete Hypothese}
    \item gerichtete Version: Benennung des genauen numerischen Zusammenhangs
    \item gerichtete Hypothesen: stärkere Aussage, schwerer zu zeigen,\\
      immer noch kein "`Beweis"'
  \end{itemize}
\end{frame}

\begin{frame}
  {Typ I- und Typ II-Fehler}
  \begin{itemize}[<+->]
    \item je nach
      \begin{itemize}[<+->]
	\item \alert{Stärke des Effekts}
	\item \alert{Homogenität} der Grundgesamtheit\slash Stichprobe
	\item \alert{Stichprobengröße}
      \end{itemize}
    \item Gefahr zweier möglicher Inferenzfehler:
      \begin{itemize}[<+->]
	\item \alert{Typ I ($\alpha$)}: H0 wird fälschlicherweise zurückgewiesen
	\item \alert{Typ II ($\beta$)}: H0 fälschlicherweise \alert{nicht} zurückgewiesen
      \end{itemize}
    \item Typ II-Fehler mehr gefürchtet, weil Artikel dann nicht angenommen wird
    \item inhaltlich beide gleich schwerwiegend
  \end{itemize}
\end{frame}

\begin{frame}
  {$\alpha$-Niveau und Teststärke}
  \begin{itemize}
    \item Logik der Hypothesenprüfung (genauer):\\
      Das Testergebnis ist so unwahrscheinlich per Zufall (H0) zu erzielen,\\
      dass H1 plausibel ist.
    \item die akzeptierte Schranke: \alert{$\alpha$-Niveau}
    \item Linguistik\slash Sozialwissenschaften: \alert{$\alpha=0.05$}
    \item wenn erreicht, gilt: \alert{Wenn in der GG der Zusammenhang nicht besteht,\\
      würde man nur in einer von 20 Stichproben die beobachtete Verteilung erwarten.}
      \Zeile
    \item \textit{If I tell you, you have a 5\% chance of being shot when you walk through the door, you go through the window.}
      \Zeile
    \item $\alpha$=Häufigkeit, mit der man sich langfristig positiv irrt\\
      (H0 korrekt, aber zurückgewiesen)
    \item vs.\ \alert{Teststärke}: $\beta$=Häufigkeit, mit der man sich langfristig negativ irrt\\
      (H0 falsch, aber nicht zurückgewiesen)
  \end{itemize}
\end{frame}

\subsection{Validität}

\begin{frame}
  {Statistische Validität}
  Gefahren für Typ I\slash II-Fehler
  \begin{itemize}[<+->]
    \item math.\ Bedingungen für Test nicht erfüllt (Typ I)
    \item kumulierter $\alpha$-Fehler (Typ I, s.\ nächste Folie)
    \item kleine Stichprobe (Typ II)
    \item zu große Variantion in der GG (Typ II)
      \Zeile
    \item in Korpora: schlechte Zusammensetzung des Korpus\\
      $\Rightarrow$ Phänomen mit mangelhafter Dispersion (Typ I)
  \end{itemize}
\end{frame}

%\begin{frame}
%  {$\alpha$-Kumulation}
%  \begin{itemize}[<+->]
%    \item Haupt-Hypothese H zerlegt in mehrere Teilhypothesen H1a, H1b, H1c
%    \item getestet mit gleichem Test an gleichem Datensatz
%    \item bei $\alpha$=0.05:
%      \begin{itemize}[<+->]
%	\item P(bestätige H1a|H0)=0.05
%	\item P(bestätige H1b|H0)=0.05
%	\item P(bestätige H1c|H0)=0.05
%	\item \alert{P(bestätige H1|H0)=\\
%	  P(bestätige H1a|H0)+P(bestätige H1b|H0)+P(bestätige H1c|H0)=0.15}
%      \end{itemize}
%    \item anders gesagt: Jede einzelne H1 wird bei $\alpha=0.05$ in einem von 20 Fällen fälschlicherweise bestätigt, aber für die gesamte H1 haben wir drei Testfälle produziert.
%    \item $\alpha$-Fehlerwahrscheinlichkeit steigt also auf $0.15$\\
%      \alert{einer von 6.67 Tests mit Typ I-Fehler}
%  \end{itemize}
%\end{frame}

\begin{frame}
  {Interne Validität}
  \begin{itemize}[<+->]
    \item Irrtum beim Herstellen des Kausalzusammenhangs
    \item Grund (Korpuslinguistik): verzerrte Stichprobenzusammensetzung
    \item Bsp.:
      \begin{itemize}[<+->]
	\item H1: Im DECOW2012 kommt öfter das Pronomen \textit{son} vor als im DWDS Kernkorpus, weil es erst nach 2000 zum eigenständigen Pronomen wurde.
	\item H0 wird auf Basis zweier Stichproben (DECOW2012, DWDS) zurückgewiesen.
	\item wirkliche Ursache: Registerunterschiede
      \end{itemize}
  \end{itemize}
\end{frame}

\begin{frame}
  {Validität des Konstrukts}
  \begin{itemize}[<+->]
    \item Korrektheit des theoretischen Konstrukts
    \item eigentlich aus der Psychologie
    \item aber riesiges mißachtetes Problem in der Linguistik
    \item Bsp.:
      \begin{itemize}
	\item Beobachtung: Das Deutsche bewahrt\\
	  genus-typische Pluralflexion am Substantiv.
	\item Konstrukt: Nominalklammer\slash Klammerprinzip\\
	  (NP-Kongruenzklammer Artikel--Substantiv, \citealt{Ronneberger2010})
	\item Hypothese zu Beobachtung: Flexionserhalt stärkt Klammerprinzip
	\item \alert{Das Konstrukt ist hochgradig beliebig und unterdefiniert.}
	\item \alert{Abhilfe: nur Konstrukte\slash Hypothesen,\\
	  die starke Vorhersagen generieren (s.\ Junggrammatiker)}
      \end{itemize}
  \end{itemize}
\end{frame}

\begin{frame}
  {Externe Validität}
  \begin{itemize}[<+->]
    \item Generalisierbarkeit der Ergebnisse (über Raum, Zeit usw.)
    \item Problem: zu große Homogenität der Stichprobe\\
      (was für statistische Validität wiederum gut ist)
    \item Bezug auf Korpora:
      \begin{itemize}[<+->]
	\item zu spezifische Stratifikation (DeReKo)
	\item verzerrte Stichprobe (evtl.\ traditionelle Webkorpora)
      \end{itemize}
  \end{itemize}
\end{frame}

\subsection{Ableitung des Fisher-Exakt-Tests aus ersten Prinzipien}

\begin{frame}
  {Ronald A.\ Fisher (1890--1962)}
  \begin{itemize}[<+->]
    \item Statistik als Teil der rationalen wissenschaftlichen Argumentation,\\
      der Interpretation von Experimenten
    \item daher: möglichst kein Mathematik-Jargon
    \item eingeschränkte Induktion als theoriegeleitete Dateninterpretation
    \item Kontrolle aller unabhängigen Variablen
    \item \alert{alle anderen (Stör-)Variablen konzeptuell zufallsgebunden}
  \end{itemize}
\end{frame}

\begin{frame}
  {Das Tassen-Problem}
  \begin{itemize}[<+->]
    \item Behauptung: Dame X kann am Geschmack erkennen,\\
      ob der Tee oder die Milch zuerst in die Tasse gegossen wurde.
    \item prä-fishersches Konzept: \alert{alle Störvariablen kontrollieren}\\
      und gleich machen, sonst keine valide Inferenz möglich
    \item Fisher: Das ist prinzipiell unmöglich, umständlich, teuer, \alert{unnötig}!
    \item wenn alle irrelvanten Stör-Faktoren zufällig, dann:
      \begin{itemize}[<+->]
	\item Variiere die relevante unabhängige Variable.
	\item Vergleiche das Ergebnis mit zufällig erwartbaren Ergebnissen.
	\item \alert{Wie unwahrscheinlich ist das erzielte Ergebnis\\
	  unter der Zufallsannahme?}
      \end{itemize}
  \end{itemize}
\end{frame}

\begin{frame}
  {Wahrscheinlichkeiten, richtig zu raten}
  \begin{itemize}[<+->]
    \item acht Tassen (zwei Milch zuerst, zwei Tee zuerst)
    \item \alert{Mit wie vielen richtigen Treffern wären Sie zufrieden?}
      \pause
      \pause
    \item Es muss die Wahrscheinlichkeit errechnet werden,\\
      eine, zwei, drei oder vier Tassen richtig zu raten.
      \Zeile
    \item \alert{Typischerweise schätzen Menschen solche Kombinatorikprobleme\\
      intuitiv falsch ein.}
  \end{itemize}
\end{frame}

\begin{frame}
  {GENAU Richtig durch Zufall}
  \begin{equation}
    P(\mathsf{richtig\ per\ Zufall})=\frac{\mathsf{Anzahl\ richtiger\ Zuweisungen}}{\mathsf{Anzahl\ aller\ potentiellen\ Zuweisungen}}
  \end{equation}
  \pause
  \begin{itemize}[<+->]
    \item Anzahl richtiger Zuweisungen: 1
    \item mögliche Zuweisungen: einfaches kombinatorisches Problem
  \end{itemize}
\end{frame}

\begin{frame}
  {mögliche Zuweisung von acht Tassen zu Milch\slash Tee zuerst}
  \begin{itemize}[<+->]
    \item erste MZ-Tasse: eine von 8
    \item zweite MZ-Tasse: eine von 7
    \item dritte MZ-Tasse: eine von 6
    \item vierte MZ-Tasse: eine von 5
    \item fünfte MZ-Tasse: STOPP (automatisch TZ)
    \item \alert{Möglichkeiten, 4 Tassen aus 8 auszuwählen:\\
      $8\cdot7\cdot6\cdot5=1680$}
  \end{itemize}
\end{frame}

\begin{frame}
  {Reihenfolge egal}
  \begin{itemize}[<+->]
    \item bisher: jedes Set von 4 MZ-Tassen ist in verschiedenen Reihenfolgen\\
      in der Menge der Möglichkeiten
    \item \alert{Möglichkeiten, vier Tassen zu ordnen:\\
      $4\cdot3\cdot2\cdot1=24$}
      \Zeile
    \item Reihenfolge hier egal, also:
  \end{itemize}
  \pause
  \begin{equation}
    \mathsf{Anzahl\ aller\ potentiellen\ Zuweisungen}=\frac{1680}{24}=70
  \end{equation}
\end{frame}

\begin{frame}
  {Wenn Dame X also genau richtig liegt}
  \begin{itemize}[<+->]
    \item per Zufall genau richtig:\\
      in einem von 70 Fällen, $p=0.014$
      \Zeile
    \item $\alpha$-Niveau von 0.05 also erreicht
  \end{itemize}
\end{frame}

\begin{frame}
  {Binomialkoeefizienten}
  \begin{itemize}[<+->]
    \item eigentlich \alert{Binomialkoeffizient}
    \item "`Lotto-Kombinationen"': $k$ aus $n$\\
      ohne Zurücklegen und ohne Beachtung der Reihenfolge
  \end{itemize}
  \Zeile
  \pause
  \begin{equation}
    \binom{n}{k}=\frac{n!}{k!(n-k)!}
  \end{equation}
\end{frame}

\begin{frame}
  {Kombinatorik für 3 richtige MZ}
  \begin{itemize}[<+->]
    \item die drei richtigen: $\binom{4}{3}$
    \item die eine falsche aus vier TZ: $\binom{4}{1}$
    \item Anzahl der Möglichkeiten drei richtige aus vier MZ\\
      und dann eine falsche aus vier TZ zu ziehen: $\binom{4}{3}\cdot\binom{4}{1}=4\cdot4=16$
  \end{itemize}
  \Zeile
  \pause
  \begin{equation}
    P(drei\ richtig\ per\ Zufall)=\frac{16}{70}=0.229
  \end{equation}
\end{frame}

\begin{frame}
  {Problem zu kleiner Stichproben}
  \begin{itemize}[<+->]
    \item \alert{Bei $\alpha=0.05$ reichen also drei richtige nicht im Ansatz!}
    \item alle schlechteren Ergebnisse: folglich auch nicht ausreichend
    \item Und bei 30 von 40 richtigen (also insgesamt 80 Tassen)?
  \end{itemize}
\end{frame}


\ifdefined\TITLE
  \section{Nächste Woche | Überblick}

  \begin{frame}
    {Einzelthemen}
    \begin{enumerate}
      \item Statistik, Inferenz und probabilistische Grammatik
      \item \alert{Deskriptive Statistik}
      \item Nichtparametrische Verfahren
      \item z-Test und t-Test
      \item ANOVA
      \item Freiheitsgrade und Effektstärken
      \item Power
      \item Lineare Modelle
      \item Generalisierte Lineare Modelle
      \item Gemischte Modelle
    \end{enumerate}
  \end{frame}
\fi

