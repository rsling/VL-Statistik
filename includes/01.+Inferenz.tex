
\section{Probative Wissenschaft}

\begin{frame}
  {Empirisch, objektiv, realistisch, exakt(?)}
  \begin{itemize}[<+->]
    \item \alert{beobachtbare} Phänomene
    \item Beobachtungen \alert{reproduzierbar}
    \item messbar = beobachtbar (Sinneswahrnehmung an sich irrelevant)
      \Zeile
    \item \alert{Realismus} | wirkliche Phänomene und ihre Mechanismen
    \item \orongsch{keine postmoderne Realitäts- und Objektivitätsverweigerung}
      \Zeile
    \item \alert{kontrolliertes Experiment}
  \end{itemize}
\end{frame}

\begin{frame}
  {Empirie | Gründe für Reproduzierbarkeitsbedingung}
  \begin{itemize}[<+->]
    \item intrinsische Ungenauigkeiten der Messung (\alert{Wirkung} plus \orongsch{Störeinflüsse})
    \item potentiell inadäquate Messung des theoretischen Konstrukts
      \Halbzeile
    \item[→] Vermeidung von Fehlschluss auf unechte Ursachen
    \item[→] \alert{relevante Ursachen}
      \Halbzeile
    \item[→] insgesamt \alert{Stärkung der Validität}
  \end{itemize}
\end{frame}

\begin{frame}
  {Beispiel | Fehlgeleitete Generative Grammatiker}
  \begin{itemize}[<+->]
    \item Gegenstand: interne (mentale) Grammatik (I-Grammatik)\\
      universeller und individueller Teil
    \item I-Grammatik bei jedem Sprecher (leicht) verschieden
    \item I-Grammatik erlaubt immer binäre Grammatikalitätsentscheidung
      \Halbzeile
    \item[→] \orongsch{Linguisten können eigene I-Grammatik untersuchen (Introspektion)!}
  \end{itemize}
  \Zeile
  \onslide<+->
  \centering 
  \rot{Das Ergebnis ist die aktuelle Krise der Linguistik.}
\end{frame}


\begin{frame}
  {Inferenz I | Positivismus und Induktion}
  \onslide<+->
  \onslide<+->
  \begin{block}
    {(Logischer) Positivismus}
    \alert{Formale} Ableitung von Wissen (= Theorien) aus Beobachtbarem und irgendeiner Logik. \alert{Induktion}. Keine Metaphysik. Keine Kreativität erwünscht. \grau{(\citealt{Carnap1928}, \ldots)} 
  \end{block}
  \Zeile
  \onslide<+->
  Aber suchen wir wirklich nur nach \alert{Mustern}, \zB in Korpusdaten?\\
    \Halbzeile
    \begin{itemize}[<+->]
      \item Was ist der \alert{zugrundeliegende Mechanismus}?
      \item Wie kommen wir zu \alert{erklärenden Theorien} von Mustern in Daten?
      \item \alert{Datenaufbereitung} (\zB im Korpus) kann dann nicht theoriegeleitet sein.
        \Halbzeile
      \item \orongsch{Die ART folgt auch nicht einfach so aus Daten!}
    \end{itemize}
\end{frame}

\begin{frame}
  {Inferenz II | Rationalistischer Probativismus (Falsifikationismus)}
  \onslide<+->
  \onslide<+->
  \begin{block}
    {Rationalistischer Probativismus}
    Theorien werden aufgestellt von \alert{Menschen, die die Welt beobachten}. Theorien werden getestet an Daten, aber nicht logisch aus Daten abgeleitet. \grau{(\citealt{Popper1962,Mayo1996}, \ldots)}
  \end{block}
  \onslide<+->
  \Zeile
  Unter dieser Philosophie werden plötzlich Dinge wichtig \ldots
    \Halbzeile
    \begin{itemize}[<+->]
      \item Ist eine \alert{Stichprobe repräsentativ} für das, was man zeigen will?
      \item Welche \alert{Methode der statistischen Analyse} wird verwendet?
      \item Für eine Korpusstudie muss die Datenaufbereitung damit theoriegeleitet sein!
      \item Liefert die Studie \alert{\textit{a serious Argument from Error}}?\\
        \onslide<+->
        \Halbzeile
        {\small\textit{There is evidence an error is absent to the extent that a \alert{procedure with\\
        a very high capability of signalling the error}, if and only if it is present,\\
      nevertheless detects no error.} \grau{\citep[16]{Mayo2018}}}\\
    \end{itemize}
\end{frame}



\begin{frame}
  {Hypothesen}
  Die konkrete Hypothesen, die in einem Experiment getestet werden,\\
  sind \orongsch{nie} die Primärhypothesen der Theorie.
  \Zeile
  \begin{itemize}[<+->]  
    \item \alert{abgeleitete Partikularhypothesen} über konkrete Erwartungen im Experiment
    \item Einfluss zahlreicher \alert{Auxiliarhypothesen}, \zB über Messprozeduren
    \item[ ] \grau{\citet{Duhem1914,Quine1951,Laudan1990}}
    \Zeile
    \item "`interessante"' Hypothesen
      \begin{itemize}[<+->]
        \item Formulierung relevanter \alert{Kausationsbedingung} (wenn, dann)
        \item \alert{universelle Gültigkeit} | ein Sprecher vs.\ alle Sprecher
        \item also \zB\ \orongsch{uninteressant} | \textit{Welchen Kasus nimmt wegen?}
      \end{itemize}
      \Halbzeile
  \end{itemize}
\end{frame}

\begin{frame}
  {Hypothesenprüfung | Probativismus (Falsifikationismus)}
  \begin{center}
    \alert{Kann die Hypothese weiter angenommen werden,\\
    oder liefert das Experiment starke Evidenz gegen sie?}
  \end{center}
  \Zeile
  \begin{itemize}[<+->]
    \item Probleme bei Prüfung
      \Viertelzeile
      \begin{itemize}[<+->]
        \item falsch abgeleitete Partikularhypothese
        \item falsche Sekundärhypothesen
	\item Störeinflüsse, intrinsische Messungenauigkeit
        \item mangelhafte \alert{Operationalisierung}
	\item zu wenige Daten (oder zu viele Daten?)
      \end{itemize}
  \end{itemize}
\end{frame}


\section{Elemente der Empirie}

\begin{frame}
  {Grundgesamtheit}
  \begin{itemize}[<+->]
    \item von Interesse | \alert{allgemeine Gesetzmäßigkeiten}
    \item also Untersuchungsgegenstand: \alert{alle x} (Sprecher, Sätze, \dots)
    \item untersuchbar | kleine Menge von x
  \end{itemize}
  \Zeile
  \centering 
  \alert{Grundgesamtheit} | alle x\\
  \Viertelzeile
  \alert{datengenerierender Prozess (DGP)} | Prozess, der \alert{alle x} hervorbringt\\
  \Zeile
  \alert{Stichprobe} | eine kleine Menge x, aus der auf Grundgesamtheit\\
  bzw.\ DGP geschlossen werden soll
\end{frame}

\begin{frame}
  {Stichprobe}
    \begin{block}{uniform zufällige Stichprobe}
      jedes Element der Grundgesamtheit hat die gleiche Chance beim Ziehen
    \end{block}
    \Zeile
    \begin{block}{stratifizierte Stichprobe}
      Stichprobe so zusammengesetzt, dass wichtige Eigenschaften proportional repräsentiert sind
    \end{block}
    \Zeile
  \begin{itemize}
    \item Problem bei Letzterem: haufenweise Auxiliarhypothesen
  \end{itemize}
\end{frame}

\begin{frame}
  {Operationalisierung und Auxiliarhypothesen}
  \begin{itemize}[<+->]
    \item \alert{Operationalisierung} | präzise Formulierung der Messmethode\\
      für ein theoretisches Konstrukt
      \Zeile
    \item Bsp.\ Konstrukt "`Satzlänge"': Wortanzahl? Phonemanzahl? Phrasenanzahl?
    \item Bsp.\ Konstrukt "`Satztopik"': Oha!?! \citep{CookBildhauer2013}
      \Zeile
    \item alle genannten Beispiele: \alert{abhängig von Auxiliarhypothesen}\\
      bzw.\ anderen theoretischen Konstrukten (Wort, Phonem, Phrase, \dots)
  \end{itemize}
\end{frame}

\begin{frame}
  {Variablen I}
  \begin{itemize}[<+->]
    \item uninteressanter Typ Fragestellung | "`Wieviel Prozent X haben Eigenschaft A?"'
    \item \orongsch{Fehlen jeglicher Aussagen über kausale Zusammenhänge}
    \item Bsp.\ | Wie oft wird \textit{wegen} mit Dat bzw.\ Gen verwendet?
      \Halbzeile
    \item Besser | "`Wie bedingt Eigenschaft B die Wahrscheinlichkeit von A bei X?"'
    \item Bsp.\ | Per Hypothese nehmen denominale Präpositionen eher den Gen als den Dat.
  \end{itemize}
  \pause
  \Halbzeile
  \begin{center}
    konzeptuell:
    \scalebox{0.7}{
      \begin{tabular}[h!]{|c||c|c|}
	\cline{2-3}
	\multicolumn{1}{c||}{} & denominale P & andere P \\
	\hline
	\hline
	Dat & $x_1$ & $x_2$\\
	\hline
	Gen & $x_3$ & $x_4$\\
	\hline
      \end{tabular}
    }
  \end{center}
\end{frame}

\begin{frame}
  {Variablen II}
  Operationalisierte und gemessene Eigenschaften sind \alert{Variablen}.\\
  \Zeile
  \begin{itemize}[<+->]
    \item im Experiment:
      \Halbzeile
      \begin{itemize}[<+->]
	\item \alert{kontrolliere} für Theorie irrelevante Variablen (\alert{Störvariablen})
          \Viertelzeile
	\item \alert{variiere} "`Ursachen-Variablen"' (\alert{unabhängige Variablen})
          \Viertelzeile
	\item \alert{beobachte} "`Wirkung-Variablen"' (\alert{abhängige Variablen})
      \end{itemize}
  \end{itemize}
\end{frame}

\begin{frame}
  {Experiment und Quasi-Experiment}
  \begin{itemize}[<+->]
    \item Problem in Astronomie, Korpuslinguistik usw. | keine Experimente möglich
    \item \alert{unabhängige Variablen nicht variierbar}
    \item Daten liegen bereits vor bzw.\ fallen vom Himmel
    \item Auswahl von Datensätzen, so dass von den unabhängigen Variablen\\
      die zur Theorieprüfung nötigen Permutationen im Datensatz vorkommen
    \item dabei Zusatzproblem bei Korpuslinguistik: Korpus meist nicht\\
      das eigene, wenig Informationen über mögliche Verzerrungen
      \Zeile
    \item \alert{Was ist die Grundgesamtheit bzw.\ der DGP?}
  \end{itemize}
\end{frame}


\section{Validität}

\begin{frame}
  {Statistische Validität}
  Gefahren für \orongsch{statistische Schlussverfahren}\\
  \Zeile
  \begin{itemize}[<+->]
    \item \alert{mathematische Vorbedingungen} für das Testverfahren nicht
    \item \alert{zu viele Partikulartests} einer übergeordneten Hypothese aus denselben Daten
    \item zu \alert{kleine Stichprobe}
    \item zu große Variation in der Grundgesamtheit
      \Zeile
    \item bei Korpora | schlechte Zusammensetzung des Korpus
  \end{itemize}
\end{frame}

\begin{frame}
  {Interne Validität}
  \begin{itemize}[<+->]
    \item Irrtum beim \orongsch{Herstellen des Kausalzusammenhangs}
    \Zeile
    \item Fiktives Bsp.:
      \Viertelzeile
      \begin{itemize}[<+->]
	\item Hypothese | Im DECOW2012 kommt öfter das Pronomen \textit{son} vor als im\\
          DWDS Kernkorpus, weil es erst nach 2000 zum eigenständigen Pronomen wurde.
	\item Die Hypothese wird bestätigt anhand von Stichproben aus den beiden Korpora.
        \item \rot{Die wirkliche Ursache sind aber Registerunterschiede}.
      \end{itemize}
  \end{itemize}
\end{frame}

\begin{frame}
  {Validität des Konstrukts}
  \begin{itemize}[<+->]
    \item Korrektheit des \alert{theoretischen Konstrukts}
      \Zeile
    \item eigentlich aus der Psychologie
    \item aber riesiges Problem in der Linguistik
      \Zeile
    \item Echtes Bsp.
      \Viertelzeile
      \begin{itemize}
	\item Beobachtung | Das Deutsche bewahrt genus-typische Pluralflexion am Substantiv.
	\item Konstrukt | Nominalklammer\slash Klammerprinzip (NP-Kongruenzklammer Art -- Subst)\\
          \grau{\citep{Ronneberger2010}}
	\item Hypothese zu Beobachtung | Flexionserhalt stärkt Klammerprinzip
          \Halbzeile
	\item \rot{Das Konstrukt ist hochgradig beliebig und unterdefiniert, damit nicht testbar.}
         \Halbzeile 
	\item \alert{Abhilfe: nur Konstrukte\slash Hypothesen, die starke Vorhersagen generieren}
      \end{itemize}
  \end{itemize}
\end{frame}

\begin{frame}
  {Externe Validität}
  \begin{itemize}[<+->]
    \item \alert{Generalisierbarkeit der Ergebnisse} (über Raum, Zeit usw.)
      \Zeile
    \item Problem | \alert{zu große Homogenität der Stichprobe}\\
      (was für statistische Validität wiederum gut ist)
      \Halbzeile
    \item Bezug auf Korpora:
      \begin{itemize}[<+->]
	\item zu spezifische Stratifikation (DeReKo)
	\item verzerrte Stichprobe (Webkorpora)
      \end{itemize}
  \end{itemize}
\end{frame}

\section{Ronald A.\ Fisher, Wahrscheinlichkeit, Ereignisraum, Teetassen}

\begin{frame}
  {Ronald A.\ Fisher (1890--1962)}
  \begin{itemize}[<+->]
    \item Statistik als Teil der rationalen wissenschaftlichen Argumentation,\\
      der Interpretation von Experimenten
    \item daher: möglichst kein Mathematik-Jargon
    \item eingeschränkte Induktion als theoriegeleitete Dateninterpretation
    \item Kontrolle aller unabhängigen Variablen
    \item \alert{alle anderen (Stör-)Variablen konzeptuell zufallsgebunden}
  \end{itemize}
\end{frame}

\begin{frame}
  {Das Tassen-Problem}
  \begin{itemize}[<+->]
    \item Behauptung: Dame X kann am Geschmack erkennen,\\
      ob der Tee oder die Milch zuerst in die Tasse gegossen wurde.
    \item prä-fishersches Konzept: \alert{alle Störvariablen kontrollieren}\\
      und gleich machen, sonst keine valide Inferenz möglich
    \item Fisher: Das ist prinzipiell unmöglich, umständlich, teuer, \alert{unnötig}!
    \item wenn alle irrelvanten Stör-Faktoren zufällig, dann:
      \begin{itemize}[<+->]
	\item Variiere die relevante unabhängige Variable.
	\item Vergleiche das Ergebnis mit zufällig erwartbaren Ergebnissen.
	\item \alert{Wie unwahrscheinlich ist das erzielte Ergebnis\\
	  unter der Zufallsannahme?}
      \end{itemize}
  \end{itemize}
\end{frame}

\begin{frame}
  {Wahrscheinlichkeiten, richtig zu raten}
  \begin{itemize}[<+->]
    \item acht Tassen (zwei Milch zuerst, zwei Tee zuerst)
    \item \alert{Mit wie vielen richtigen Treffern wären Sie zufrieden?}
      \pause
      \pause
    \item Es muss die Wahrscheinlichkeit errechnet werden,\\
      eine, zwei, drei oder vier Tassen richtig zu raten.
      \Zeile
    \item \alert{Typischerweise schätzen Menschen solche Kombinatorikprobleme\\
      intuitiv falsch ein.}
  \end{itemize}
\end{frame}

\begin{frame}
  {GENAU Richtig durch Zufall}
  \begin{equation}
    P(\mathsf{richtig\ per\ Zufall})=\frac{\mathsf{Anzahl\ richtiger\ Zuweisungen}}{\mathsf{Anzahl\ aller\ potentiellen\ Zuweisungen}}
  \end{equation}
  \pause
  \begin{itemize}[<+->]
    \item Anzahl richtiger Zuweisungen: 1
    \item mögliche Zuweisungen: einfaches kombinatorisches Problem
  \end{itemize}
\end{frame}

\begin{frame}
  {mögliche Zuweisung von acht Tassen zu Milch\slash Tee zuerst}
  \begin{itemize}[<+->]
    \item erste MZ-Tasse: eine von 8
    \item zweite MZ-Tasse: eine von 7
    \item dritte MZ-Tasse: eine von 6
    \item vierte MZ-Tasse: eine von 5
    \item fünfte MZ-Tasse: STOPP (automatisch TZ)
    \item \alert{Möglichkeiten, 4 Tassen aus 8 auszuwählen:\\
      $8\cdot7\cdot6\cdot5=1680$}
  \end{itemize}
\end{frame}

\begin{frame}
  {Reihenfolge egal}
  \begin{itemize}[<+->]
    \item bisher: jedes Set von 4 MZ-Tassen ist in verschiedenen Reihenfolgen\\
      in der Menge der Möglichkeiten
    \item \alert{Möglichkeiten, vier Tassen zu ordnen:\\
      $4\cdot3\cdot2\cdot1=24$}
      \Zeile
    \item Reihenfolge hier egal, also:
  \end{itemize}
  \pause
  \begin{equation}
    \mathsf{Anzahl\ aller\ potentiellen\ Zuweisungen}=\frac{1680}{24}=70
  \end{equation}
\end{frame}

\begin{frame}
  {Wenn Dame X also genau richtig liegt}
  \begin{itemize}[<+->]
    \item per Zufall genau richtig:\\
      in einem von 70 Fällen, $p=0.014$
      \Zeile
    \item $\alpha$-Niveau von 0.05 also erreicht
  \end{itemize}
\end{frame}

\begin{frame}
  {Binomialkoeefizienten}
  \begin{itemize}[<+->]
    \item eigentlich \alert{Binomialkoeffizient}
    \item "`Lotto-Kombinationen"': $k$ aus $n$\\
      ohne Zurücklegen und ohne Beachtung der Reihenfolge
  \end{itemize}
  \Zeile
  \pause
  \begin{equation}
    \binom{n}{k}=\frac{n!}{k!(n-k)!}
  \end{equation}
\end{frame}

\begin{frame}
  {Kombinatorik für 3 richtige MZ}
  \begin{itemize}[<+->]
    \item die drei richtigen: $\binom{4}{3}$
    \item die eine falsche aus vier TZ: $\binom{4}{1}$
    \item Anzahl der Möglichkeiten drei richtige aus vier MZ\\
      und dann eine falsche aus vier TZ zu ziehen: $\binom{4}{3}\cdot\binom{4}{1}=4\cdot4=16$
  \end{itemize}
  \Zeile
  \pause
  \begin{equation}
    P(drei\ richtig\ per\ Zufall)=\frac{16}{70}=0.229
  \end{equation}
\end{frame}

\begin{frame}
  {Problem zu kleiner Stichproben}
  \begin{itemize}[<+->]
    \item \alert{Bei $\alpha=0.05$ reichen also drei richtige nicht im Ansatz!}
    \item alle schlechteren Ergebnisse: folglich auch nicht ausreichend
    \item Und bei 30 von 40 richtigen (also insgesamt 80 Tassen)?
  \end{itemize}
\end{frame}


\ifdefined\TITLE
  \section{Nächste Woche | Überblick}

  \begin{frame}
    {Einzelthemen}
    \begin{enumerate}
      \item Statistik, Inferenz und probabilistische Grammatik
      \item \alert{Deskriptive Statistik}
      \item Nichtparametrische Verfahren
      \item z-Test und t-Test
      \item ANOVA
      \item Freiheitsgrade und Effektstärken
      \item Power
      \item Lineare Modelle
      \item Generalisierte Lineare Modelle
      \item Gemischte Modelle
    \end{enumerate}
  \end{frame}
\fi

