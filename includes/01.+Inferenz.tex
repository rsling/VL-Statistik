
\section{Probative Wissenschaft}

\begin{frame}
  {Empirisch, objektiv, realistisch, exakt(?)}
  \begin{itemize}[<+->]
    \item \alert{Beobachtbare} Phänomene
    \item Beobachtungen \alert{reproduzierbar}
    \item Messbar = beobachtbar (Sinneswahrnehmung an sich irrelevant)
      \Zeile
    \item \alert{Realismus} | wirkliche Phänomene und ihre Mechanismen
    \item \orongsch{Keine postmoderne Realitäts- und Objektivitätsverweigerung}
      \Zeile
    \item \alert{Kontrolliertes Experiment}
  \end{itemize}
\end{frame}

\begin{frame}
  {Empirie | Gründe für Reproduzierbarkeitsbedingung}
  \begin{itemize}[<+->]
    \item Intrinsische Ungenauigkeiten der Messung (\alert{Wirkung} plus \orongsch{Störeinflüsse})
    \item Potentiell inadäquate Messung des theoretischen Konstrukts
      \Halbzeile
    \item[→] Vermeidung von Fehlschluss auf unechte Ursachen
    \item[→] \alert{Relevante Ursachen}
      \Halbzeile
    \item[→] Insgesamt \alert{Stärkung der Validität}
  \end{itemize}
\end{frame}

\begin{frame}
  {Beispiel | Fehlgeleitete Generative Grammatiker}
  \begin{itemize}[<+->]
    \item Gegenstand: interne (mentale) Grammatik (I-Grammatik)\\
      universeller und individueller Teil
    \item I-Grammatik bei jedem Sprecher (leicht) verschieden
    \item I-Grammatik erlaubt immer binäre Grammatikalitätsentscheidung
      \Halbzeile
    \item[→] \orongsch{Linguisten können eigene I-Grammatik untersuchen (Introspektion)!}
  \end{itemize}
  \Zeile
  \onslide<+->
  \centering 
  \rot{Das Ergebnis ist die aktuelle Krise der Linguistik.}
\end{frame}


\begin{frame}
  {Inferenz I | Positivismus und Induktion}
  \onslide<+->
  \onslide<+->
  \begin{block}
    {(Logischer) Positivismus}
    \alert{Formale} Ableitung von Wissen (= Theorien) aus Beobachtbarem und irgendeiner Logik. \alert{Induktion}. Keine Metaphysik. Keine Kreativität erwünscht. \grau{(\citealt{Carnap1928}, \ldots)} 
  \end{block}
  \Zeile
  \onslide<+->
  Aber suchen wir wirklich nur nach \alert{Mustern}, \zB in Korpusdaten?\\
    \Halbzeile
    \begin{itemize}[<+->]
      \item Was ist der \alert{zugrundeliegende Mechanismus}?
      \item Wie kommen wir zu \alert{erklärenden Theorien} von Mustern in Daten?
      \item \alert{Datenaufbereitung} (\zB im Korpus) kann dann nicht theoriegeleitet sein.
        \Halbzeile
      \item \orongsch{Die ART folgt auch nicht einfach so aus Daten!}
    \end{itemize}
\end{frame}

\begin{frame}
  {Inferenz II | Rationalistischer Probativismus (Falsifikationismus)}
  \onslide<+->
  \onslide<+->
  \begin{block}
    {Rationalistischer Probativismus}
    Theorien werden aufgestellt von \alert{Menschen, die die Welt beobachten}. Theorien werden \alert{getestet an Daten}, aber nicht logisch aus Daten abgeleitet. Wissenschaft lernt aus Fehlern. \grau{(\citealt{Popper1962,Mayo1996}, \ldots)}
  \end{block}
  \onslide<+->
  \Zeile
  Unter dieser Philosophie werden plötzlich Dinge wichtig \ldots
    \Halbzeile
    \begin{itemize}[<+->]
      \item Ist eine \alert{Stichprobe repräsentativ} für das, was man zeigen will?
      \item Welche \alert{Methode der statistischen Analyse} wird verwendet?
      \item Für eine Korpusstudie muss die Datenaufbereitung damit theoriegeleitet sein!
      \item Liefert die Studie \alert{\textit{a serious Argument from Error}}?\\
        \onslide<+->
        \Halbzeile
        {\small\textit{There is evidence an error is absent to the extent that a \alert{procedure with\\
        a very high capability of signalling the error}, if and only if it is present,\\
      nevertheless detects no error.} \grau{\citep[16]{Mayo2018}}}\\
    \end{itemize}
\end{frame}



\begin{frame}
  {Hypothesen}
  Die konkrete Hypothesen, die in einem Experiment getestet werden,\\
  sind \orongsch{nie} die Primärhypothesen der Theorie.
  \Zeile
  \begin{itemize}[<+->]  
    \item \alert{Abgeleitete Partikularhypothesen} über konkrete Erwartungen im Experiment
    \item Einfluss zahlreicher \alert{Auxiliarhypothesen}, \zB über Messprozeduren
    \item[ ] \grau{\citet{Duhem1914,Quine1951,Laudan1990}}
    \Zeile
    \item "`Interessante"' Hypothesen
      \begin{itemize}[<+->]
        \item Formulierung relevanter \alert{Kausationsbedingung} (wenn, dann)
        \item \alert{Universelle Gültigkeit} | ein Sprecher vs.\ alle Sprecher
        \item Also \zB\ \orongsch{uninteressant} | \textit{Welchen Kasus nimmt wegen?}
      \end{itemize}
      \Halbzeile
  \end{itemize}
\end{frame}

\begin{frame}
  {Hypothesenprüfung | Probativismus (Falsifikationismus)}
  \begin{center}
    \alert{Kann die Hypothese weiter angenommen werden,\\
    oder liefert das Experiment starke Evidenz gegen sie?}
  \end{center}
  \Zeile
  \begin{itemize}[<+->]
    \item Probleme bei Prüfung
      \Viertelzeile
      \begin{itemize}[<+->]
        \item Falsch abgeleitete Partikularhypothese
        \item Falsche Sekundärhypothesen
	\item Störeinflüsse, intrinsische Messungenauigkeit
        \item Mangelhafte \alert{Operationalisierung}
	\item Zu wenige Daten (oder zu viele Daten?)
      \end{itemize}
  \end{itemize}
\end{frame}


\section{Elemente der Empirie}

\begin{frame}
  {Grundgesamtheit}
  \begin{itemize}[<+->]
    \item Von Interesse | \alert{allgemeine Gesetzmäßigkeiten}
    \item Also Untersuchungsgegenstand: \alert{alle x} (Sprecher, Sätze, \dots)
    \item Untersuchbar | kleine Menge von x
  \end{itemize}
  \Zeile
  \centering 
  \alert{Grundgesamtheit} | alle x\\
  \Viertelzeile
  \alert{Datengenerierender Prozess (DGP)} | Prozess, der \alert{alle x} hervorbringt\\
  \Zeile
  \alert{Stichprobe} | eine kleine Menge x, aus der auf Grundgesamtheit\\
  bzw.\ DGP geschlossen werden soll
\end{frame}

\begin{frame}
  {Stichprobe}
    \begin{block}{Uniform zufällige Stichprobe}
      Jedes Element der Grundgesamtheit hat die gleiche Chance beim Ziehen.
    \end{block}
    \Zeile
    \begin{block}{Stratifizierte Stichprobe}
      Die Stichprobe ist so zusammengesetzt, dass wichtige Eigenschaften proportional repräsentiert sind.
    \end{block}
    \Zeile
  \begin{itemize}
    \item Problem bei Letzterem: haufenweise Auxiliarhypothesen
  \end{itemize}
\end{frame}

\begin{frame}
  {Operationalisierung und Auxiliarhypothesen}
  \begin{itemize}[<+->]
    \item \alert{Operationalisierung} | präzise Formulierung der Messmethode\\
      für ein theoretisches Konstrukt
      \Zeile
    \item Bsp.\ Konstrukt "`Satzlänge"': Wortanzahl? Phonemanzahl? Phrasenanzahl?
    \item Bsp.\ Konstrukt "`Satztopik"': Oha!?! \citep{CookBildhauer2013}
      \Zeile
    \item Alle genannten Beispiele \alert{abhängig von Auxiliarhypothesen}\\
      bzw.\ anderen theoretischen Konstrukten (Wort, Phonem, Phrase, \dots)
  \end{itemize}
\end{frame}

\begin{frame}
  {Variablen I}
  \begin{itemize}[<+->]
    \item Uninteressanter Typ Fragestellung | "`Wieviel Prozent X haben Eigenschaft A?"'
    \item \orongsch{Fehlen jeglicher Aussagen über kausale Zusammenhänge}
    \item Bsp.\ | Wie oft wird \textit{wegen} mit Dat bzw.\ Gen verwendet?
      \Halbzeile
    \item Besser | "`Wie bedingt Eigenschaft B die Wahrscheinlichkeit von A bei X?"'
    \item Bsp.\ | Per Hypothese nehmen denominale Präpositionen eher den Gen als den Dat.
  \end{itemize}
  \pause
  \Halbzeile
  \begin{center}
    Konzeptionell:
    \scalebox{0.7}{
      \begin{tabular}[h!]{|c||c|c|}
	\cline{2-3}
	\multicolumn{1}{c||}{} & denominale P & andere P \\
	\hline
	\hline
	Dat & $x_1$ & $x_2$\\
	\hline
	Gen & $x_3$ & $x_4$\\
	\hline
      \end{tabular}
    }
  \end{center}
\end{frame}

\begin{frame}
  {Variablen II}
  Operationalisierte und gemessene Eigenschaften sind \alert{Variablen}.\\
  \Zeile
  \begin{itemize}[<+->]
    \item Im Experiment:
      \Halbzeile
      \begin{itemize}[<+->]
	\item \alert{Kontrolliere} für Theorie irrelevante Variablen (\alert{Störvariablen})\\
          bzw.\ verlass dich auf deren Zufallsverteilung (Fisher, s.\,u.).
          \Viertelzeile
	\item \alert{Variiere} "`Ursachen-Variablen"' (\alert{unabhängige Variablen}).
          \Viertelzeile
	\item \alert{Beobachte} "`Wirkung-Variablen"' (\alert{abhängige Variablen}).
      \end{itemize}
  \end{itemize}
\end{frame}

\begin{frame}
  {Experiment und Quasi-Experiment}
  \begin{itemize}[<+->]
    \item Problem in Astronomie, Korpuslinguistik usw. | keine Experimente möglich
    \item \alert{Unabhängige Variablen nicht variierbar}
    \item Daten liegen bereits vor bzw.\ fallen vom Himmel
    \item Auswahl von Datensätzen, so dass von den unabhängigen Variablen\\
      die zur Theorieprüfung nötigen Permutationen im Datensatz vorkommen
    \item Dabei Zusatzproblem bei Korpuslinguistik: Korpus meist nicht\\
      das eigene, wenig Informationen über mögliche Verzerrungen
      \Zeile
    \item \alert{Was ist die Grundgesamtheit bzw.\ der DGP?}
  \end{itemize}
\end{frame}


\section{Validität}

\begin{frame}
  {Statistische Validität}
  Gefahren für \orongsch{statistische Schlussverfahren}\\
  \Zeile
  \begin{itemize}[<+->]
    \item \alert{Falsches Testverfahren} für die gegebene Situation
    \item \alert{Mathematische Vorbedingungen} für das Testverfahren nicht
    \item \alert{Zu viele Partikulartests} einer übergeordneten Hypothese aus denselben Daten
    \item Zu \alert{kleine Stichprobe}
    \item Zu \alert{große Stichprobe}
    \item Zu große Variation in der Grundgesamtheit
  \end{itemize}
\end{frame}

\begin{frame}
  {Interne Validität}
  \begin{itemize}[<+->]
    \item Irrtum beim \orongsch{Herstellen des Kausalzusammenhangs}
    \Zeile
    \item Fiktives Bsp.:
     \Halbzeile 
      \begin{itemize}[<+->]
        \item Korpora | DWDS-Kernkorpus enthält Texte 1900--2000, DECOW12 Texte nach 2000
          \Viertelzeile
	\item Hypothese | Im DECOW12 kommt öfter das Pronomen \textit{son} vor als im\\
          DWDS Kernkorpus, weil es erst nach 2000 zum eigenständigen Pronomen wurde.
          \Viertelzeile
	\item Die Hypothese wird bestätigt anhand von Stichproben aus den beiden Korpora.
          \Viertelzeile
        \item \rot{Die wirkliche Ursache sind aber Registerunterschiede}.
      \end{itemize}
  \end{itemize}
\end{frame}

\begin{frame}
  {Validität des Konstrukts}
  \begin{itemize}[<+->]
    \item Korrektheit des \alert{theoretischen Konstrukts}
      \Zeile
    \item Eigentlich aus der Psychologie
    \item Aber riesiges Problem in der Linguistik
      \Zeile
    \item Echtes Bsp.
      \Viertelzeile
      \begin{itemize}
	\item Beobachtung | Das Deutsche bewahrt genus-typische Pluralflexion am Substantiv.
	\item Konstrukt | Nominalklammer\slash Klammerprinzip (NP-Kongruenzklammer Art -- Subst)\\
          \grau{\citep{Ronneberger2010}}
	\item Hypothese (post-hoc zur Beobachtung) | Flexionserhalt stärkt Klammerprinzip
          \Halbzeile
	\item \rot{Das Konstrukt ist hochgradig beliebig und unterdefiniert, damit nicht testbar.}
         \Halbzeile 
	\item \alert{Abhilfe: nur Konstrukte\slash Hypothesen, die starke Vorhersagen generieren}
      \end{itemize}
  \end{itemize}
\end{frame}

\begin{frame}
  {Externe Validität}
  \begin{itemize}[<+->]
    \item \alert{Generalisierbarkeit der Ergebnisse} (über Raum, Zeit usw.)
      \Zeile
    \item Problem | \alert{zu große Homogenität der Stichprobe}\\
      (was für statistische Validität wiederum gut ist)
      \Halbzeile
    \item Bezug auf Korpora:
      \begin{itemize}[<+->]
	\item Zu spezifische Stratifikation (DeReKo)
	\item Verzerrte Stichprobe (Webkorpora)
      \end{itemize}
  \end{itemize}
\end{frame}

\section{Ronald A.\ Fisher, Wahrscheinlichkeit, Ereignisraum, Teetassen}

\begin{frame}
  {Ronald A.\ Fisher (1890--1962)}
  \begin{itemize}[<+->]
    \item Statistik als Teil der rationalen wissenschaftlichen Argumentation,\\
      der Interpretation von Experimenten
      \Viertelzeile
    \item Möglichst kein Mathematik-Jargon, eher intuitiv zugängliche\\
      mathematische Konzepte
      \Halbzeile
    \item \orongsch{Eingeschränkte statistische Inferenz als theoriegeleitete Dateninterpretation}
      \Halbzeile
    \item Kontrolle aller unabhängigen Variablen
    \item \alert{Alle anderen (Stör-)Variablen konzeptuell zufallsgebunden}
  \end{itemize}
\end{frame}

\begin{frame}
  {The Tea-Tasting Lady}
  Muriel Bristow behauptet, sie könne am Geschmack einer Tasse Tee erkennen,\\
  ob die Milch oder der Tee zuerst eingeschenkt wurde. Fisher führt\\
  ein Experiment durch (acht Tassen, vier mit dem Tee zuerst) und fragt,\\
  wie wir entscheiden können, ob das Ergebnis davon zeugt, dass sie\\
  diese Fähigkeit wirklich hat.\\
  \Halbzeile
  \begin{itemize}[<+->]
    \item \gruen{Liegt das Ergebnis \orongsch{deutlich} über dem per Zufall erwartbaren Niveau?}
      \Halbzeile
    \item Idee vor Fisher | \alert{alle Störvariablen kontrollieren}\\
      und gleich machen, dann ist induktive Inferenz möglich 
    \item Fisher | Das ist prinzipiell unmöglich, umständlich, teuer und \alert{unnötig}!
      \Halbzeile
    \item Wenn alle irrelevanten Störvariablen zufallsverteilt sind, dann gilt:
      \Viertelzeile
      \begin{itemize}[<+->]
	\item Variiere die relevante unabhängige Variable.
          \Viertelzeile
        \item Vergleiche das Ergebnis mit \alert{zufällig erwartbaren Ergebnissen}.
      \end{itemize}
  \end{itemize}
\end{frame}

\begin{frame}
  {Wahrscheinlichkeit}
  Bayesische Wahrscheinlichkeit (\textit{inverse probability)}\\
  \begin{itemize}[<+->]
    \item \rot{Für wie wahrscheinlich hält Individuum I das Ereignis E?}
    \item Subjektiv, berücksichtigt vorherige Überzeugung
    \item Aktualisierung von Überzeugungen
    \item Basiert auf \alert{Bayes Rule} (Tomas Bayes 1763)
    \item \alert{Ereignisraum} (s.\,u.) irrelevant!
  \end{itemize}
  \Zeile
  Frequentistische Wahrscheinlichkeit\\
  \begin{itemize}[<+->]
    \item \alert{Wie viele mögliche Ereignisse e\Sub{i} aus E treten ein?}
    \item Zu jedem Experiment gehört ein \alert{Ereignisraum} (s.\,u.)!
    \item Daher \alert{objektiv}, unabhängig von Überzeugungen
    \item \alert{Wenn ein Ereignis e\Sub{i} eingetreten ist, wird seine Wahrscheinlichkeit uninteressant.}
    \item Geeignet für rationalistisch-probativistische Wissenschaftsphilosophie
  \end{itemize}
\end{frame}

\begin{frame}
  {Frequentismus | Ereignisraum (sample space)}
  Für ein Experiment gilt:\\
  \Halbzeile
  \begin{itemize}[<+->]
    \item Wir beobachten \alert{$n$ Messungen} (Stichprobe),\\
      jede Messung wird aus einer \alert{definierten Menge von möglichen Messungen}.
      \Halbzeile
      \begin{itemize}[<+->]
        \item Bsp. | 10 Mal einen Würfel werfen \{1, 2, 3, 4, 5, 6\}.
          \Viertelzeile
        \item Bsp. | Je 10 Akzeptabilitätsurteile unter 2 Bedingungen von 100 Probanden \{Ja, Nein\}.
      \end{itemize}
      \Zeile
    \item Wir bekommen ein \alert{konkretes Ergebnis}.
      \Halbzeile
      \begin{itemize}[<+->]
        \item Bsp. | 8 von 10 Würfen mehr als drei Augen.
          \Viertelzeile
        \item Bsp. | "`Mehr"' Ja-Antworten unter Bedingung A (schon deutlich komplexeres Design).
      \end{itemize}
      \Halbzeile
    \item Wir müssen berücksichtigen, \alert{wie viele Ergebnisse (und welche) es insgesamt\\
      hätte geben können}, um zu bewerten, wie unwahrscheinlich das Ergebnis \rot{war}.
      \Viertelzeile
    \item \alert{Ereignisraum (sample space)} | Menge der möglichen Ausgänge des Experiments
  \end{itemize}
\end{frame}

\begin{frame}
  {Warum "`\rot{war}"'?}
  \grau{Wir müssen berücksichtigen, wie viele Ergebnisse (und welche) es insgesamt\\
    hätte geben können, um zu bewerten, wie unwahrscheinlich das Ergebnis \rot{war}.}\\
    \Zeile
    \begin{itemize}[<+->]
      \item Jedes eingetretene Ereignis hat die Wahrscheinlichkeit 1.
        \Halbzeile
        \begin{itemize}[<+->]
          \item Die Wahrscheinlichkeit, dass Helmut Kohl 1998 abgewählt wurde, ist 1.
            \Viertelzeile
          \item Die Wahrscheinlichkeit, dass wir 8 Würfe mit mehr als drei Augen hatten, ist 1.
        \end{itemize}
      \Halbzeile
      \item \rot{Nach dem Experiment} | P(konkreter Ausgang des Experiments wurde erzielt) = 1
      \item \gruen{Vor dem Experiment} | P(konkreter Ausgang des Experiments wird erzielt werden) < 1
    \end{itemize}
    \Zeile
    \centering 
    \orongsch{Die übelsten Fehler in der Bewertung statistischer Ergebnisse rühren daher,\\
    dass Menschen diese Sachverhalte vergessen.}\\
\end{frame}

\begin{frame}
  {Zurück zur Tea-Tasting Lady}
  \alert{Design des Experiments} | Muriel Bristow probiert \alert{acht Tassen},\\
  (vier mit Milch zuerst, vier mit Tee zuerst) und \alert{wählt die vier mit Tee zuerst aus}.\\
  \Zeile
  \begin{itemize}[<+->]
    \item \alert{Mit wie vielen richtigen Treffern wären Sie zufrieden?}
      \Halbzeile
    \item Es muss die \alert{frequentistische Wahrscheinlichkeit} errechnet werden,\\
      eine, zwei, drei oder vier Tassen auch \orongsch{per Zufall} richtig zu raten.
      \Halbzeile
    \item Dann können wir beurteilen, ob das Ergebnis\\
      \textit{\orongsch{deutlich} \gruen{über dem erwartbaren Niveau}} liegt.
  \end{itemize}
\end{frame}

\begin{frame}
  {Ausgang 1 | Alle vier Tassen korrekt}
  Allgemein:\\
  \Viertelzeile
  \begin{equation}
    P(\mathsf{konkreter\ Ausgang})=\frac{\gruen{\mathsf{Anzahl\ richtiger\ Zuweisungen}}}{\alert{\mathsf{Anzahl\ aller\ potentiellen\ Zuweisungen}}}
  \end{equation}\\
  \Zeile
  Für diesen Ausgang:\\
  \Viertelzeile
  \begin{equation}
    P(\mathsf{vier\ Tassen\ korrekt})=\frac{\gruen{1}}{\alert{\mathsf{Anzahl\ aller\ potentiellen\ Zuweisungen}}}
  \end{equation}
\end{frame}

\begin{frame}
  {Wie viele Möglichkeiten gibt es?}
  Wir wählen \alert{vier \textit{Tee zuerst}-Tassen (TZ) aus acht Tassen} aus:\\
  \Halbzeile
  \begin{itemize}[<+->]
    \item erste TZ-Tasse: eine von 8 (bleiben 7)
    \item zweite TZ-Tasse: eine von 7 (bleiben 6)
    \item dritte TZ-Tasse: eine von 6 (bleiben 5)
    \item vierte TZ-Tasse: eine von 5 (bleiben 4)
      \Halbzeile
    \item[→] \rot{STOPP} | alle anderen 4 Tassen automatisch MZ
  \end{itemize}
  \Zeile
  \centering 
  Also naiv gedacht | \alert{$8\cdot7\cdot6\cdot5=1680$}
\end{frame}

\begin{frame}
  {Die Reihenfolge}
  \orongsch{1680 ist zu hoch}, denn je nachdem, welche Tasse aus den verbleibenden wir wählen,\\
  ergeben sich \orongsch{andere Permutationen (Reihenfolgen) desselben Ergebnisses}.\\
  \Zeile
  \begin{itemize}[<+->]
    \item Bsp. | Auswahl von Tasse 7, 3, 6, 1 identisch zu 3, 1, 6, 7 usw.
      \Halbzeile
    \item Es gibt von jeder möglichen Auswahl gleich viele Permutationen.
    \item Und zwar die Anzahl der \alert{Möglichkeiten, vier Tassen zu ordnen: $4\cdot3\cdot2\cdot1$}
      \Zeile
  \end{itemize}
  \Zeile
  \begin{equation}
    \mathsf{Anzahl\ aller\ potentiellen\ Zuweisungen}=\frac{8\cdot7\cdot6\cdot5}{4\cdot3\cdot2\cdot1}=\frac{1680}{24}=70
  \end{equation}
\end{frame}

\begin{frame}
  {Wenn sie also genau richtig liegt \ldots}
  Wahrscheinlichkeit, per Zufall genau richtig zu liegen:\\
  \Viertelzeile
  \begin{equation}
    P(\mathsf{vier\ Tassen\ korrekt})=\frac{\gruen{1}}{\alert{70}}=0.014
  \end{equation}\\
  \Zeile
  \centering 
  \blau{Interpretieren Sie das Ergebnis.}
\end{frame}

\begin{frame}
  {Für die Generalisierung der Berechnung}
  \begin{itemize}[<+->]
    \item Eigentlich haben wir es mit \alert{Binomialkoeffizienten} zu tun.
      \Halbzeile
    \item "`Lotto-Kombinationen"' | $k$ aus $n$\\
      \alert{ohne Zurücklegen und ohne Beachtung der Reihenfolge}
  \end{itemize}
  \Zeile
  \pause
  \begin{equation}
    \binom{n}{k}=\frac{n!}{k!(n-k)!}
  \end{equation}
\end{frame}

\begin{frame}
  {Ausgang 2 | Drei Tassen korrekt}
  Berechnung mit dem Binomialkoeffizienten
  \Halbzeile
  \begin{itemize}[<+->]
    \item Die drei richtigen aus vier TZ| \gruen{$\binom{4}{3}$}
    \item Die eine falsche aus vier TZ | \rot{$\binom{4}{1}$}
  \end{itemize}
  \Zeile
  \pause
  \begin{equation}
    P(drei\ richtig\ per\ Zufall)=\frac{\gruen{\binom{4}{3}}\cdot\rot{\binom{4}{1}}}{70}=\frac{16}{70}=\alert{0.229}
  \end{equation}\\
  \Zeile
  \centering 
  \blau{Interpretieren Sie das Ergebnis.}
\end{frame}

\begin{frame}
  {Darstellung als Kreuztabelle}
  \centering 
  \alert{Ausgang 1}\\
  \begin{tabular}[h]{rrcc}
    \toprule
    && \multicolumn{2}{c}{\textbf{Realität}} \\
    && \textbf{Tee zuerst} & \textbf{Milch zuerst} \\
    \midrule
    \multirow{2}{*}{\textbf{Lady}} & \textbf{Tee zuerst} & 4 & 0 \\
    & \textbf{Milch zuerst} & 0 & 4 \\
    \bottomrule
  \end{tabular}\\
  \vspace{2\baselineskip}
  \alert{Ausgang 2}\\
  \begin{tabular}[h]{rrcc}
    \toprule
    && \multicolumn{2}{c}{\textbf{Realität}} \\
    && \textbf{Tee zuerst} & \textbf{Milch zuerst} \\
    \midrule
    \multirow{2}{*}{\textbf{Lady}} & \textbf{Tee zuerst} & 3 & 1 \\
    & \textbf{Milch zuerst} & 1 & 3 \\
    \bottomrule
  \end{tabular}
\end{frame}

\begin{frame}
  {Stichprobengröße und Effektstärke}
  \begin{itemize}[<+->]
    \item Unbefriedigendes Ergebnis bei 3 von 4 richtigen Tassen
      \Halbzeile
    \item Sehr \orongsch{kleine Stichprobe} | nur perfektes Ergebnis zufriedenstellend
    \item \orongsch{Effektstärke} | Vielleicht kann MB ca.\ 75 \% aller Tassen richtig erkennen.
  \end{itemize}
  \Zeile
  \centering 
  Bei größerer Stichprobe | Was ist mit \alert{30 von 40 richtigen Tassen},\\also insgesamt 80 Tassen?\\
  \Halbzeile
  Das wäre die \alert{gleiche Effektstärke}, aber eine \alert{größere Stichprobe}.
\end{frame}

\begin{frame}
  {Letzte Warnung}
  Was zeigt man mit so einem Experiment? Und was nicht?\\
  \Zeile
  \begin{itemize}[<+->]
    \item Der Ausgang \rot{war} \gruen{ziemlich unwahrscheinlich},\\
      bevor das Experiment durchgeführt wurde.
    \item Daher gehen wir bis auf Weiteres davon aus, \gruen{dass ein Effekt vorliegt} \ldots
    \item \ldots\ \rot{oder zufällig ein seltenes Ereignis eingetreten ist!!!}
      \Halbzeile
    \item Wenn Sie mit den Geburtsdaten Ihrer Familie im Lotto gewinnen,\\
      \rot{ist ein seltenes Ereignis eingetreten}, Sie haben aber \rot{nicht gezeigt,\\
      dass Ihre Geburtsdaten die Lottokugeln beeinflussen}!
      \Halbzeile
    \item \alert{Ein solches Ergebnis beweist also nichts!}
    \item Die Logik basiert auf der Annahme einer wiederholten Testung.
    \item[→] Wenn wir das Experiment \alert{sehr oft} machen, und es gibt \alert{keinen Effekt},\\
      dann nähert sich die \alert{Verteilung der Ergebnisse der Zufallsverteilung} an.
  \end{itemize}
\end{frame}

\ifdefined\TITLE
  \section{Nächste Woche | Überblick}
  \begin{frame}
    {Einzelthemen}
    \begin{enumerate}
      \item Inferenz 
      \item \alert{Deskriptive Statistik}
      \item Nichtparametrische Verfahren
      \item z-Test und t-Test
      \item ANOVA
      \item Freiheitsgrade und Effektstärken
      \item Power und Severity
      \item Lineare Modelle
      \item Generalisierte Lineare Modelle
      \item Gemischte Modelle
    \end{enumerate}
  \end{frame}
\fi

